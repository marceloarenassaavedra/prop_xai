

Unfortunately, computing small $\delta$-SRs is computationally challenging, even when attempting to find approximate solutions. Let us contextualize our main result by summarizing first what is known about the complexity of computing $\delta$-SRs and their deterministic predecessors, $1$-SRs.  

\citet{NEURIPS2020_b1adda14} showed that computing a minimum $1$-SR is $\Sigma_2^p$-hard for neural networks, $\np$-hard for decision trees, and polynomial-time solvable for linear models. Then, \citet[Theorem 2.4]{Waldchen_MacDonald_Hauch_Kutyniok_2021} showed that computing minimum $\delta$-SRs for neural networks is hard for $\np^{\mathrm{PP}}$, and \citet{NEURIPS2022_b8963f6a} proved that even for the restricted class of decision trees, which are usually considered interpretable, minimum $\delta$-SRs cannot be computed in polynomial time unless $\ptime = \np$ (and neither can subset-minimal $\delta$-SRs for $\delta < 1$, in contrast to the $\delta = 1$ setting which is in $\ptime$~\cite{izzaExplainingDecisionTrees2020,roaModelInterpretabilityLens2020}). 
For linear models, even computing the value
\[
    \Pr_{\vz \sim \U{\vy}}\big[\Lin(\vz) = \Lin(\vx)\big]
\]
exactly is $\shp$-hard~\cite{NEURIPS2020_b1adda14}, from where the following is easy to show.\footnote{\citet{izzaComputingProbabilisticAbductive2023} already made a more general observation of this form, but did not provide a hardness result for linear models.}
 
\begin{proposition}\label{prop:hardness}
    Given a linear model $\Lin$, an instance
    $\vx$, and $\delta \in [0,1]$, the value $\opt(\Lin, \vx, \delta)$
    cannot be computed in polynomial time unless $\fptime = \shp$.
\end{proposition}

Furthermore, the situation does not improve if we aim to efficiently approximate the value $\opt(\M, \vx, \delta)$. \citet[Theorem 2.5]{Waldchen_MacDonald_Hauch_Kutyniok_2021} studied general classifiers (e.g., neural networks) and showed that no algorithm can achieve an approximation factor of $d^{1-\alpha}$ for this problem, where $d$ is the dimension of the classifier and $\alpha > 0$, unless $\ptime = \np$. \citet{Kozachinskiy_2023} proved that this approximation task is also hard for decision trees.

% As the notion of approximation at play is subtle, let us clarify the precise inapproximability statements we are referring to.

% \begin{theorem}
% Unless $\ptime = \np$,  

% \end{theorem}

%  In light of these negative results, in what follows we consider a relaxed notion of $\delta$-minSR where $\delta$ can be replaced by a closed value $\delta'$.

 
% In what follows, we prove that these negative results also extend to linear models, as it is not possible to compute minimum $\delta$-SRs for such models in polynomial time unless $\fptime = \shp$,\footnote{$\fptime$ is the class of all functions that can be computed in polynomial time.} which is widely believed to be false.
% \begin{proposition}\label{prop:hardness}
% Given a linear model $\Lin$, an instance
% $\vx$, and $\delta \in [0,1]$, the value $\opt(\Lin, \vx, \delta)$
% cannot be computed in polynomial time unless $\fptime = \shp$.
% \end{proposition}
% Unfortunately, the situation does not improve if we aim to efficiently approximate the value $\opt(\Lin, \vx, \delta)$. \citet{Waldchen_MacDonald_Hauch_Kutyniok_2021} showed that no algorithm can achieve an approximation factor of $d^{1-\alpha}$ for this problem, where $d$ is the dimension of the classifier and $\alpha > 0$, unless $\ptime = \np$. Additionally, \citet{Kozachinskiy_2023} proved that this approximation task is also hard for decision trees. In light of these negative results, in what follows we consider a relaxed notion of $\delta$-minSR where $\delta$ can be replaced by a closed value $\delta'$.
% \begin{definition}[\cite{Waldchen_MacDonald_Hauch_Kutyniok_2021}]
%         Given $\delta, \epsilon \in [0, 1]$, a $(\delta, \epsilon)$-sufficient reason ($(\delta, \epsilon)$-SR) for an instance $\vx$, is a partial instance $\vy \subseteq \vx$ such that
%         \[
%                 \Pr_{\vz \sim \U{\vy}}\big[\M(\vz) = \M(\vx)\big] \geq \delta.
%         \]
%         \label{def:delta-SR}
% \end{definition}






%Notably, none of these hardness results rely on the distribution $\D$ being complicated, as they were proved taking $\D = \textrm{Uniform}(\{0, 1\}^d)$. 
%In stark contrast, we will be able to provide positive approximability results in terms of $\delta$ for linear models, which is, to the best of our knowledge, the first positive theoretical result in the area.
%Let us formally define the computational problem at hand, assuming the uniform distribution over $\{0, 1\}^d$, which we will simply denote by $\mathcal{U}$ from now on.
However, these hardness results do not preclude the existence of
efficient algorithms for computing or approximating $\delta$-SR for
linear models. Hence, the goal of this section is to explore these
questions for such models, given their practical importance.

\subsection{A Simple Relaxation: $(\delta, \varepsilon)$-min-SR}

In light of the hardness results for $\delta$-SRs, it is natural to consider a further relaxation that would allow for tractability. Consider for instance a customer of a bank who wants a $0.95$-SR for why their application for a loan was rejected. Such an explanation would consist of a small number of features of their application profile that are relevant to the decision since $95\%$ of applicants with such a profile would also get rejected. We expect that, in such a scenario, the user would not particularly care if the explanation she obtains holds 
for $95\%$ of potential applicants or for $94.9997\%$ of them. In other words, the value of $\delta$ is chosen in a trade-off between the size of the explanation and the desired level of confidence or ``explanation power''. We posit that in such a trade-off, the user is more sensitive to increases in the explanation size than they are to a minor perturbation in $\delta$, the probability guarantee. As we showed in~\Cref{prop:delta-sr-size}, by changing $\delta$ very slightly, the size of the best explanation can change significantly. This motivates the following definition:
% \begin{itemize}
%     \item \textbf{Approximation in terms of $\delta$}: Given a linear model $\Lin$, an instance $\vx$, and a value $\delta$, we want to compute a $\delta'$-SR of size $k^\star(\Lin, \vx, \delta')$ such that $\delta'$ is close to $\delta$. Intuitively, this corresponds to the idea of stakeholders not caring about the exact value of $\delta$; e.g., $90\%$ of completions of $\vy$ agree with $\vx$ has roughly the same human implications as $89.9997\%$.
%     \item \textbf{Approximation in terms of $k^\star$}: Given a linear model $\Lin$, an instance $\vx$, and a value $\delta$, we want to compute a $\delta$-SR of size $k$ such that $k$ is not much bigger than $k^\star(\Lin, \vx, \delta)$. Intuitively, even though stakeholders want \emph{small} explanations, it is not required to find the smallest possible explanation.
% \end{itemize}

% Furthermore, we define a slightly different sense in which the value of $\delta$ can be relaxed:

\begin{definition}[$(\delta, \varepsilon)$-min-SR]
    Given a model $\M$, an instance $\vx$, and values $\delta, \varepsilon \in (0, 1)$, we say a partial instance $\vy$ of size is a $(\delta, \varepsilon)$-min-SR if there exists a value $\delta^\star \in [\delta - \varepsilon, \delta + \varepsilon]$ such that
    $\vy$ is a minimum $\delta^\star$-SR for $\vx$ under $\M$.
\end{definition}

Note that, even though the guarantee of a $(\delta, \varepsilon)$-min-SR is symmetric around $\delta$, our definition is such that the ability of efficiently computing $(\delta, \varepsilon)$-min-SRs is enough for the following two tasks:
\begin{enumerate}
    \item A user wants an explanation as small as possible and of probability ``close'' to $\delta$. Then, by computing a~$(\delta-\varepsilon/2, \varepsilon/2)$-min-SR, they obtain an explanation whose probability guarantee is at most $\varepsilon$ away from $\delta$, and is no larger in size than the minimum $\delta$-SR.
    \item The owner of the model wants to offer a $\delta$-SR that is as small as possible to a customer, and they want to be strict on the $\delta$ part, since offering a $(\delta - \varepsilon)$-SR would be misleading and could lead to legal issues. Then, by computing a $(\delta+\varepsilon/2, \varepsilon/2)$-min-SR, they can guarantee that the explanation is at least $\delta$-SR, while still being likely much smaller than a minimum $1$-SR. 
\end{enumerate}

The inapproximability result of~\citet{Kozachinskiy_2023} can be translated to the $(\delta, \varepsilon)$-min-SR problem as follows:
\begin{theorem}[\citet{Kozachinskiy_2023}, Theorem 1]
    Unless SAT can be solved in quasi-polynomial times, one cannot compute a $(\delta, \varepsilon)$-min-SR for decision trees in polynomial time, and furthermore, any polynomial-time algorithm that guarantees to provide a $\delta'$-SR for some $\delta' \in [\delta-\varepsilon, \delta+\varepsilon]$ will produce explanations that are up to $\Omega(d^{1-\alpha})$ times larger than any $(\delta, \varepsilon)$-min-SR, for any $\alpha > 0$.
\end{theorem}
Note that this hardness result for decision trees implies in turn hardness for neural networks by using standard compilation techniques~\citep{NEURIPS2020_b1adda14}.
Our main result is that, for linear models, we can efficiently compute $(\delta, \varepsilon)$-min-SRs, making them the first class of models for which we have such a positive result. To state our runtime more cleanly, we use the standard notation $\widetilde{O}(f)$ to mean $O(f \cdot \log(f)^c)$ for some positive constant $c \in \mathbb{R}$. 
\begin{theorem}
    \label{prop:smoothed-explanation}
    Given a linear model $\Lin$ and an input $\vx$, we can compute a $(\delta, \varepsilon)$-min-SR successfully with probability $1 - \gamma$ in time $\widetilde{O}\left( \frac{d}{\varepsilon^2\gamma^2}\right)$; that is, polynomial in $d$, $1/\varepsilon$, and $1/\gamma$.
\end{theorem}

We remark that previous approaches for computing approximate probabilistic explanations lacked theoretical guarantees on the size of the explanations produced~\citep{izzaComputingProbabilisticAbductive2023,Izza2021EfficientEW,izza2024locallyminimalprobabilisticexplanations}.

% \begin{theorem}\label{thm:delta-large}
%     For any $\delta > 0.05$, we can compute a min-$(\delta, \varepsilon)$-SR in polynomial time for linear models under the uniform distribution in time
%     \[ 
%         \tilde{O}\left(\frac{d}{\varepsilon^2\gamma^2}\right).
%     \]
% \end{theorem}

% \begin{open}
%     Is there an FPRAS with respect to $k^\star$ for  Uniform-Min-$\delta$-SR$(\textsc{Linear})$?
% \end{open}

% \begin{open}
%     Is there an FPRAS with respect to $\delta$ for Product-Min$\delta$-SR$(\textsc{Linear})$? That is, under product distributions in which each feature $i$ has an associated probability $p_i$ of taking value $1$ and these are all independent.
% \end{open}

% \begin{theorem}\label{thm:k-approximation}
%     The Uniform-Min-$\delta$-SR$(\textsc{Linear})$ problem can be approximated over $k^\star$ with additive error $1$ and probability of success $1-\gamma$ in time $\tilde{O}\left( \frac{d^2}{\varepsilon^2\gamma^2}\right)$.
% \end{theorem}

In order to prove~\Cref{prop:smoothed-explanation} we will need two main ideas: first, the fact that we can estimate the probabilities of models accepting a partial instance through sampling (which is already present in the work of~\citet{izza2024locallyminimalprobabilisticexplanations}), and second, that under the uniform distribution it is easy to decide which features ought to be part of small explanations over linear models.

\subsection{Estimating the Probability of Acceptance}
 The  hardness of computing
$\Pr_{\vz  \sim \U(\vy)}[\M(\vz) = 1]$ is about computing it to arbitrarily high precision, i.e., with an additive error within $O(2^{-d})$. However, computing a less precise estimation of $\Pr_{\vz \sim \U(\vy)}[\M(\vz) = 1]$ is simple, as the next fact (which is a direct consequence of Hoeffding's inequality)  states.

\begin{fact}\label{fact:hoeffding}
    Let $f$ be an arbitrary boolean function on $n$ variables. Let $M$ be any positive integer,
    and let $\vx_1, \ldots, \vx_M$ be $M$ uniformly random samples from $\{0, 1\}^n$. Then 
    \[
        \widehat{\mu}(M) := \frac{\sum_{i=1}^M [f(\vx_i) = 1]}{M}
    \]
    is an unbiased estimator for 
    \[
        \mu := \Pr_{\vx \in \{0, 1\}^n}[f(\vx) = 1],
    \]
    and 
    \[
    \Pr[\left|\widehat{\mu}(M) - \mu \right| \leq t] \geq 1 - 2\exp(-2t^2 M),
    \]
    which is at least $1 - \gamma$ for $M = \frac{1}{2t^2} \log(2/\gamma)$.
\end{fact}

As a consequence of the previous idea, although a minimum $\delta$-SR might be hard to compute, this crucially depends on the value of $\delta$. In order to deal with this, our algorithm will sample a value $\delta^\star$ uniformly at random from $[\delta-\varepsilon, \delta+\varepsilon]$, and then compute a minimum $\delta^\star$-SR. Intuitively, the idea is that as $\delta^\star$ is chosen at random, it will be unlikely that a value that makes the computation hard is chosen. 

Before proving~\Cref{prop:smoothed-explanation}, we need to prove a lemma concerning the easiness of selecting the features of the desired explanation.

\subsection{Feature Selection}

Even if we were granted an oracle computing the probabilities
\(
    \Pr_{\vz  \in \D(\vy)}[\M(\vz) = 1]
\), that would not be necessarily enough to efficiently compute a minimum $\delta$-SR. Indeed, for decision trees, the counting problem can be easily solved in polynomial time~\cite{NEURIPS2020_b1adda14}, and yet the computation of $\delta$-SRs of minimum size is hard, even to approximate~\cite{NEURIPS2022_b8963f6a, Kozachinskiy_2023}.
Intuitively, the problem for decision trees is that, even if we were told that the minimum $\delta$-SR has exactly $k$ features, it is not obvious how to search for it better than enumerating all $\binom{d}{k}$ subsets. The case of linear models, however, is different, at least under the uniform distribution. In this case, every feature $i$ that is not part of the explanation will take value $0$ or $1$ independently with probability $\nicefrac{1}{2}$, and contribute to the classification according to its weight $w_i$. In other words, we can sort the features according to their weights (with some care about signs), and select them greedily to build a small $\delta$-SR. A proof for the deterministic case ($\delta = 1$) was already given in~\cite{NEURIPS2020_b1adda14} and sketched earlier on by~\cite{ExplainingNaiveBayes}.

\begin{definition}\label{def:scores}
    Given a linear model $\Lin = (\vw, t)$, and an instance $\vx$, both having dimension $d$, we define the \emph{score} of feature $i \in [d]$ as 
    \[
        s_i := w_i \cdot (2x_i - 1) \cdot (2\Lin(\vx) - 1).    
    \]
\end{definition}

In other words, the sign of $s_i$ is $+1$ if the feature is ``helping'' the classification, and $-1$ if it is ``hurting'' it. The magnitude of $s_i$ is proportional to the weight of the feature $i$. Changing the value of feature $i$ in an instance $\vx$ would decrease $\vw \cdot \vx$ by $s_i$ if $\Lin(\vx) = 1$, and increase it by $s_i$ if $\Lin(\vx) = 0$.
For the uniform distribution (or more generally, any distribution in which all features are Bernoulli variables with the same parameter), we can prove the following lemma that basically states that, for linear models it is good to choose features greedily according to their score.

\begin{lemma}\label{lemma:greedy}
    Given a linear model $\Lin$, and an instance $\vx$, if $\vy^{(0)}, \ldots, \vy^{(d)}$ are the partial instances of $\vx$ such that $\vy^{(k)} \subseteq \vx$ is defined only in the top $k$ features of maximum score, then
    \[ 
        \Pr_{\vz \sim \U(\vy^{(k+1)})}[\Lin(\vz) = \Lin(\vx)] \geq \Pr_{\vz \sim \U(\vy^{(k)})}[\Lin(\vz) = \Lin(\vx)]
    \]
    for all $0 \in \{1, \ldots, d-1\}$, and naturally, 
    \[ 
    \Pr_{\vz \sim \U(\vy^{(d)})}[\Lin(\vz) = \Lin(\vx)] = 1.
    \]
    Moreover, $\opt(\Lin, \vx, \delta) = k$ if and only if $\vy^{(k)}$ is a $\delta$-SR for $\vx$, and either $k = 0$ or $\vy^{(k-1)}$ is not a $\delta$-SR for $\vx$. 
    \end{lemma}

Even though a proof of~\Cref{lemma:greedy} is presented in the supplementary material, let us provide a self-contained example to help convince a reader of the veracity of the lemma.
\begin{example}\label{ex:greedy}
 Consider an instance $\vx = (1,\, 0, \, 0, \, 1, \, 1)$ and the linear model $\Lin$ be defined by 
 \[ 
    \vw = (5, \, 1, \, -3,\, 2, -1) \quad ; \quad t = 5.
 \]
 It is easy to check that $\vw \cdot \vx = 5$, and thus $\Lin(\vx) = 1$. The feature scores, according to~\Cref{def:scores}, are:
 \[
  s_1 = 5, \; s_2 = -1, \; s_3 = 3, \; s_4 = 2, \; s_5 = -1.
 \]
For the first part, the main idea is that a positive score $s_i$ means that the feature is helping the classification (i.e., adding it to a partial instance does not decrease its probability guarantee), while a negative score means that the feature is hurting the classification (i.e., adding it to a partial instance does not increase its probability guarantee). 
Because the partial instances 
\[ \vy^{(0)} \subseteq \vy^{(1)} \subseteq \cdots \vy^{(d)}
\] 
are obtained by adding a single feature at a time, and thus features are added in decreasing order of their scores, then this procedure will have two phases: (i) First, it will add features with a positive score, which raise or maintain the probability of the classification being the same as $\vx$, as the lemma says, and then (ii) it will start adding features with a negative score, which would seem to contradict the lemma, but it turns out that at that point the partial instance $\vy^{(k)}$ would have probability guarantee $1$; this is because $\vy^{(d)} = \vx$, which trivially has probability guarantee $1$.
\Cref{table:ex-greedy1} presents the probabilities associated to the partial instances $\vy^{(0)}, \ldots, \vy^{(d)}$.

For the second part, consider the partial instances $\vy^{\star} = (\bot, \, 0, \, 0, \, 1, \, 1)$ and $\vy^{\dagger} = (1, \, \bot, \, 0, \, 1, \, 1)$. The instance $\vx$ is a completion of both $\vy^{\star}$ and $\vy^{\dagger}$, but $\vy^{\star}$ also has completion 
\(
    \vx^{\star} = (0, \, 0, \, 0, \, 1, \, 1),
\)
whereas $\vy^{\dagger}$ has also completion 
\(
    \vx^{\dagger} = (1, \, 1, \, 0, \, 1, \, 1).
\)
Note that $\vw \cdot  \vx^{\star} = 1 = \vw \cdot \vx - s_1$, whereas $\vw \cdot  \vx^{\dagger} = 6 = \vw \cdot \vx - s_2$. Intuitively, this means that it is better to keep feature $1$ as part of the explanation, but not feature $2$. If we want an explanation with only two features, we should choose feature $1$ and feature $3$, as they have the highest scores. Indeed,~\Cref{table:ex-greedy2} presents the probabilities to all possible explanations of size $2$.
\begin{table}
    \caption{Table of probabilities associated to~\Cref{ex:greedy}.}\label{table:ex-greedy2}
    \centering
    \begin{tabular}{rrr}
        \toprule
        Partial instance & Features included & Probability\\ \midrule 
    $(1, \,\bot, \,0, \,\bot, \,\bot)$ & $\{1, \,3\}$ & $\nicefrac{ 7 }{ 8 }$\\
    $(1, \,\bot, \,\bot, \,1, \,\bot)$ & $\{1, \,4\}$ & $\nicefrac{ 5 }{ 8 }$\\
    $(\bot, \,\bot, \,0, \,1, \,\bot)$ & $\{3, \,4\}$ & $\nicefrac{ 1 }{ 2 }$\\
    $(\bot, \,\bot, \,0, \,\bot, \,1)$ & $\{3, \,5\}$ & $\nicefrac{ 3 }{ 8 }$\\
    $(\bot, \,0, \,0, \,\bot, \,\bot)$ & $\{2, \,3\}$ & $\nicefrac{ 3 }{ 8 }$\\
    $(1, \,\bot, \,\bot, \,\bot, \,1)$ & $\{1, \,5\}$ & $\nicefrac{ 3 }{ 8 }$\\
    $(1, \,0, \,\bot, \,\bot, \,\bot)$ & $\{1, \,2\}$ & $\nicefrac{ 3 }{ 8 }$\\
    $(\bot, \,\bot, \,\bot, \,1, \,1)$ & $\{4, \,5\}$ & $\nicefrac{ 1 }{ 4 }$\\
    $(\bot, \,0, \,\bot, \,1, \,\bot)$ & $\{2, \,4\}$ & $\nicefrac{ 1 }{ 4 }$\\
    $(\bot, \,0, \,\bot, \,\bot, \,1)$ & $\{2, \,5\}$ & $\nicefrac{ 1 }{ 8 }$\\
        \bottomrule
    \end{tabular}
\end{table}
\begin{table}
\caption{Table of probabilities associated to~\Cref{ex:greedy}. The last column denotes the score of the feature added to the partial instance in that row with respect to the previous row.}\label{table:ex-greedy1}
\centering
\begin{tabular}{rrrr}
    Partial instance & Features & Probability & Score \\ \midrule 
$\vy^{(0)}$  & $(\bot, \,\bot, \,\bot, \,\bot, \,\bot)$ & $\nicefrac{ 1 }{ 4 }$ & - \\
$\vy^{(1)}$  &$(1, \,\bot, \,\bot, \,\bot, \,\bot)$ & $\nicefrac{ 1 }{ 2 }$ & 5\\
$\vy^{(2)}$  &$(1, \,\bot, \,0, \,\bot, \,\bot)$ & $\nicefrac{ 7 }{ 8 }$ & 3\\
$\vy^{(3)}$ & $(1, \,\bot, \,0, \,1, \,\bot)$ & $\nicefrac{ 1 }{ 1 }$ & 2 \\
$\vy^{(4)}$ &$(1, \,0, \,0, \,1, \,\bot)$ & $\nicefrac{ 1 }{ 1 }$ & -1 \\
$\vy^{(5)}$ & $(1, \,0, \,0, \,1, \,1)$ & $\nicefrac{ 1 }{ 1 }$ & -1\\
\bottomrule
\end{tabular}
\end{table}
\end{example}

% Now, let us state a simple monotonicity lemma that is part of the proof of~\Cref{lemma:greedy}; Intuitively, adding features in order of decreasing scores will never decrease the probability of the explanation being correct.
% \begin{lemma}\label{lemma:monotonicity}
% Given a linear model $\Lin$, an instance $\vx$, if $\vy^{(1)}, \ldots, \vy^{{d}}$ are  partial instances of $\vx$ such that $\vy^{(k)} \subseteq \vx$ is defined only in the top $k$ features of maximum score, then
% \[ 
%     \Pr_{\vz \sim \U(\vy^{k+1})}[\Lin(\vz) = \Lin(\vx)] \geq \Pr_{\vz \sim \U(\vy^{(k)})}[\Lin(\vz) = \Lin(\vx)]
% \]
% for all $k \in \{1, \ldots, d-1\}$, and naturally, $\Pr_{\vz \sim \U(\vy^{d})}[\Lin(\vz) = \Lin(\vx)] = 1$.
% \end{lemma}
% % \todo[inline]{Bernardo: The proof of this is trivial but annoying, it suffices to do an exchange argument. I will write it down later.}

With~\Cref{lemma:greedy} in hand, we can proceed to prove~\Cref{prop:smoothed-explanation}.

\renewcommand{\algorithmiccomment}[1]{\; \; \texttt{/* #1 */}}
\begin{algorithm}[tb]
	\caption{Uni-Linear MonteCarlo}
	\label{alg:algorithm}
	\textbf{Input}: Linear model $\Lin$, instance $\vx$, parameters $\delta \in (0, 1)$\\
	\textbf{Parameter}:  $\varepsilon \in (0, 1)$,  $\gamma \in (0, 1)$\\
	\textbf{Output}: A value $\delta^\star \in [\delta-\varepsilon, \delta+\varepsilon]$ together with an optimal $\delta^\star$-SR explanation for $\vx$.\\
	\begin{algorithmic}[1] %[1] enables line numbers
	\STATE Sample $\delta^\star$ uniformly at random from $[\delta-\varepsilon, \delta + \varepsilon]$
	\STATE Compute the score $s_i$ for each feature
	\STATE Let $\vec{\mathcal{F}}$ be the sequence of pairs $(i, s_i)$ sorted decreasingly by $s_i$
	\STATE Let $\ell$ be the max value such that $\vec{\mathcal{F}}[\ell] = (\lambda, s_\lambda)$ has $s_\lambda > 0$.
	\STATE For $k \in \{1, \ldots, \ell\}$, let $\vy^{(k)} \subseteq \vx$ be the partial instance defined only in features $\vec{\mathcal{F}}[1][1], \ldots, \vec{\mathcal{F}}[k][1]$.
	\STATE Let $\textrm{LB} = 0$, and $\textrm{UB} = \ell$.
	\STATE $M \gets \frac{2\log d^2}{\varepsilon^2 \gamma^2} \log(2 \log d / \gamma)$.

	\WHILE[Binary Search]{$\textrm{LB} \neq \textrm{UB}$} 
	% \COMMENT{Binary Search}
		\STATE $m \gets \left(\textrm{LB} + \textrm{UB} \right)/2$.
		\STATE $\hat{v}_m \gets \textsc{MonteCarloSampling}(\Lin, \vy^{(m)}, M)$.
		\IF {$\hat{v}_m \geq \delta^\star$}
			\STATE $\textrm{UB} \gets m$
		\ELSE
			\STATE $\textrm{LB} \gets (m+1)$
		\ENDIF
	\ENDWHILE
	\STATE $k^\star \gets \textrm{LB}$ (or equivalently, $\textrm{UB}$)
	\STATE \RETURN $(\delta^\star, \vy^{(k^\star)})$
	\end{algorithmic}
	\end{algorithm}


\begin{algorithm}[tb]
	\caption{MonteCarloSampling}	\label{alg:montecarlo}
	\textbf{Input}: Linear model $\Lin$, instance $\vx$, number of samples $M$\\
	\textbf{Output}: An estimate~$\hat{v}$~of~$\Pr_{\vz \in \mathcal{U}_\vx}[\Lin(\vz)=1]$.\\
	\begin{algorithmic}[1]
	\STATE $\hat{v} \gets 0$
	\FOR{$i = 1$ to $M$}
		\STATE Sample $\vz \sim \textsc{Comp}(\vx)$
		\STATE $\hat{v} \gets \hat{v} + \Lin(\vz)$
	\ENDFOR
	\STATE $\hat{v} \gets \hat{v}/M$
	\STATE \RETURN $\hat{v}$
\end{algorithmic}
\end{algorithm}


	

\begin{proof}[Proof of~\Cref{prop:smoothed-explanation}]
We use~\Cref{alg:algorithm}. Let us define the values $v_k$ as
	\[
		v_k := \Pr_{\vz \in \textsc{Comp}(\vy^{(k)})}[\Lin(\vz) = 1].
	\]
As a result of the binary search (lines 8-16), the algorithm obtains $k^\star$, the smallest $k$ such that
\[
	\hat{v}_k \geq \delta^\star,
\]
and our goal is to show that with good probability $k^\star$ is also the smallest $k$ such that $v_k \geq \delta^\star$, which would imply the correctness of the algorithm.
Let $S$ be a random variable corresponding to the set of values $k$ such that~\Cref{alg:algorithm} enters line $10$ with $m=k$, and note that if for every $k$ in $S$ it happens that the events 
\[
	A_k := \left(v_k \geq \delta^\star \right) \text{ and }  B_k := \left(\hat{v_k}(M) \geq \delta^\star\right)
\]
are equivalent (i.e., either both occur or neither occurs), then the algorithm will succeed, as that would indeed imply that $k^\star$ is the smallest $k$ such that $v_k \geq \delta^\star$. 

Then, define events $E_k$ and $F_k$ as follows:
\begin{align*}
	E_k &:= |\delta^\star - v_k| > \frac{\epsilon \gamma}{\log d},\\
	F_k &:= k \in S \land |\hat{v_k}(M) - v_k| \leq \frac{\epsilon \gamma}{\log d}.
\end{align*}
We claim that if both $E_k$ and $F_k$ hold for some $k$, then $A_k$ and $B_k$ are equivalent events. Indeed,
\begin{align*}
	A_k &\iff v_k \geq \delta^\star\\
		&\iff v_k \geq \delta^\star + \frac{\epsilon \gamma}{\log d} \tag{By $E_k$}\\
		&\iff v_k - \frac{\epsilon \gamma}{\log d} \geq \delta^\star\\
		&\iff \hat{v_k}(M) \geq \delta^\star \tag{By $F_k$}\\
		&\iff B_k.
\end{align*}

Thus, if we show that $E_k$ and $F_k$ hold with good probability for every $k \in S$, we can conclude the theorem.
To see that, note first that for every $k \in [d]$, line 1 implies
\[
	\Pr[\overline{E_k}] = \Pr\left[\delta^\star \in \left[v_k - \frac{\epsilon \gamma}{\log d}, v_k + \frac{\epsilon \gamma}{\log d}\right]\right] \leq \frac{\frac{2\epsilon \gamma}{\log d}}{2\epsilon} = \frac{\gamma}{\log d}.
\]

It is tempting to say that we have 
\[ 
	\Pr\left[\bigcap_{k \in S} E_k\right] = 1 -  \Pr\left[\bigcup_{k \in S} \overline{E_k} \right] \geq 1 - \gamma,
\]
through a union bound on the $\log d$ elements of $S$\footnote{For simplicity we will say $|S| \leq \log d$, even though the exact bound for a binary search is $|S| \leq \lfloor \log d  + 1 \rfloor$.}. However, $S$ itself is a random variable, and $E_k$ and $k \in S$ are not (necessarily) independent events, so we need to be more careful. 
Using the law of total probabilities, we have 
\begin{align*}
	\Pr\left[\bigcap_{k \in S} E_k\right] &= \sum_{S' \subseteq [d]} \Pr\left[S = S'  \mid \bigcap_{k \in S'} E_k \right] \Pr\left[\bigcap_{k \in S'} E_k\right]\\
	&= \sum_{\substack{S' \subseteq [d]\\ |S'| \leq \log d}} \Pr\left[S = S' \mid \bigcap_{k \in S'} E_k\right] \Pr\left[\bigcap_{k \in S'} E_k\right],
\end{align*}
where we can now effectively use the union bound to say that for any fixed $S'$ with $|S'| \leq \log d$ we have
\[ 
	\Pr\left[\bigcap_{k \in S'} E_k \right] \geq 1 - \gamma,
\]
from where we get
\begin{align*}
	\Pr\left[\bigcap_{k \in S} E_k\right] &=  \sum_{\substack{S' \subseteq [d]\\ |S'| \leq \log d}} \Pr\left[S = S'\mid \bigcap_{k \in S'} E_k\right] \Pr\left[\bigcap_{k \in S'} E_k\right]\\
	&\geq (1-\gamma) \sum_{\substack{S' \subseteq [d]\\ |S'| \leq \log d}} \Pr\left[S = S' \mid \bigcap_{k \in S'} E_k\right]\\
	&= 1 - \gamma.
\end{align*}
 Let us now argue that $F_k$ holds with good probability for every $k \in S$. Indeed, we have for any $k$ that
 \begin{align*}
 \Pr\left[ \, \overline{F_k} \mid k \in S \right] &= \Pr\left[|\hat{v_k}(M) - v_k| > \frac{\epsilon \gamma}{\log d}\right]\\
 	&\leq \frac{\gamma}{\log d}, \tag{By~\Cref{fact:hoeffding} and line 7.}
 \end{align*}
from where a final union bound (using the same trick as for $E_k$) yields 
\begin{align*}
	\Pr\left[\bigcap_{k \in S} F_k\right] &= 1 - \Pr\left[\bigcup_{k \in S} \overline{F_k}\right]\\
	&\geq 1 - \sum_{k \in S} \Pr\left[\overline{F_k} \mid k \in S\right]\\
	&\geq 1 - \sum_{k \in S} \frac{\gamma}{\log d}\\
	&= 1 - \gamma.
\end{align*}
Therefore, the algorithm will succeed with probability at least 
\[ 
	\Pr\left[\bigcap_{k \in S} E_k\right] \cdot \Pr\left[\bigcap_{k \in S} F_k\right] \geq (1-\gamma)^2 \geq 1-2\gamma.
\]
The runtime is simply
$O(\log d \cdot M \cdot d  )$; as (i) the binary search performs $O(\log d)$ steps; (ii) each of the binary search steps requires $M$ samples, and (iii) each sample requires evaluating the model $\Lin$ and thus takes time $O(d)$. Naturally, running the algorithm with $\gamma' = 1/2 \cdot \gamma$ will yield a success probability of $1-\gamma$ without changing the asymptotic runtime, and thus we conclude the proof.


% We thus want to lower bound the probability of some decently large value $\Delta$ such that $|\delta^\star - v_k| \geq \Delta$ for every $k \in [d]$; that way, it will be enough to estimate the values $v_k$ up to an additive error of $\Delta/2$. Given that there are $d$ values $v_
% k$, and each of them forbids a segment of size $2\Delta$ for $\delta^\star$, then the probability of choosing a value of $\delta^\star$ such that $|\delta^\star - v_k| \geq \Delta$ for every $k \in [d]$ is at least
% \[
% 	1 - \frac{d\Delta}{\epsilon}.
% \]
% As we want our algorithm to succeed with probability $1 - \gamma$, that requires 
% \(
% \frac{d\Delta}{\epsilon} \leq \gamma
% \)
% and thus
% \[
% 	\Delta \leq \frac{\epsilon \gamma}{d}.
% \]
% Thus we need to sample with an additive error smaller than
% \(
% \frac{\epsilon}{2d \gamma},
% \)
% which requires the number of samples $M$ to be such that
% \[
% 	\frac{\log M}{\sqrt{M}} \leq \frac{\epsilon\gamma}{2d}.
% \] 
% %%Considering that
% %%\[
% %%	\frac{\log M}{\sqrt{M}} \leq \frac{\log M}{\log M \sqrt[2+\gamma]{M}} = \frac{1}{\sqrt[2+\gamma]{M}},
% %%\]
% %Let us try to simplify things by considering establishing
% %\[
% %	\frac{1}{\sqrt{M'}} \leq \frac{\epsilon\gamma}{2d}, 
% %\]
% %or equivalently,
% %\[
% %	M \geq \left( \frac{2d}{\epsilon\gamma} \right)^2.
% %\]
% Now we pose
% \[
% 	M =	9\cdot \left( \frac{2d}{\epsilon\gamma} \right)^2\log\left(\frac{2d}{\epsilon\gamma}\right)^2,
% \]
% so that $\log M \leq 3\log\left(\frac{2d}{\epsilon\gamma}\right)$, and thus
% \[
% 	\frac{\log M}{\sqrt{M}} \leq \frac{3\log(2/(\epsilon\gamma))}{3\left( \frac{2d}{\epsilon\gamma} \right)\log\left(\frac{2d}{\epsilon\gamma}\right)} = \frac{\epsilon\gamma}{2d},
% \]
% as desired.

% Thus, the algorithm above, which requires a sampling process at most $\log d$ times, using binary search to find the min, can be carried out with probability of success at least $1-\gamma$. in time 
% \[
% O(\log d \cdot M) = \tilde{O}\left(\left(\frac{d}{\epsilon\gamma}\right)^2\right).
% \]


% \begin{figure}
% 	\centering
% 	\scalebox{0.8}{
% 	\begin{tikzpicture}
% 		\node[] (beg) at (0, 0) {};
% 		\node[] (end) at (11, 0) {};
% 				\node[] (a) at (10, 0.5) {\small $1$};
% 		\node[] (amark) at (10, 0) {\small $\mid$};
% 		\node[] (b) at (5, 0.5) {\small $\nicefrac{1}{2}$};
		
% 		\node[] (b) at (0, 0.5) {\small $0$};
		
% 		\node[] (bmark) at (5, 0) {\small $\mid$};
		
% 		\node[] (leftPar) at (5.5, 0) {\textcolor{blue}{$\Big[$}};
% 		\node[] (rightPar) at (9.5, 0) {\textcolor{blue}{$\Big]$}};
		
% 		\draw[-, blue, dashed, very thick]  (leftPar) -- (rightPar);
		
% 		\draw[|-, very thick] (beg) -- (leftPar);
% 			\draw[->, very thick] (rightPar) -- (end);

		
% 		\node[] (left) at (5.5, -0.7) {\textcolor{blue}{\scriptsize $\delta-\varepsilon$}};
% 		\node[] (right) at (9.5, -0.7) {\textcolor{blue}{\scriptsize $\delta+\varepsilon$}};
		
% 		\node[] (delta) at (7.5, -0.5) {\textcolor{blue}{$\delta$}};
		
% 		\node[] (dmark) at (7.5, 0) {\textcolor{blue}{\small $\mid$}};
		
		
% 		\node[] (v1) at (8.8, 0) {\textcolor{purple}{\small $\mid$}};
% 		\node[] (v1) at (8.8, -0.4) {\textcolor{purple}{\scriptsize $v_1$}};
		
% 		\node[] (v2) at (6.6, 0) {\textcolor{purple}{\small $\mid$}};
% 		\node[] (v2) at (6.6, -0.4) {\textcolor{purple}{\scriptsize $v_2$}};
		
% 		\node[] (v4) at (3.6, 0) {\textcolor{purple}{\small $\mid$}};
% 		\node[] (v4) at (3.6, -0.4) {\textcolor{purple}{\scriptsize $v_3$}};
		
% 		\node[] (v3) at (1.2, 0) {\textcolor{purple}{\small $\mid$}};
% 		\node[] (v3) at (1.2, -0.4) {\textcolor{purple}{\scriptsize $v_4$}};
		
% 		\draw[rectangle, fill=purple, line width=0pt] (6.3, -0.2) -- (6.3, 0.2) -- (6.9, 0.2) -- (6.9, -0.2) -- (6.3, -0.2);
		
% 		\draw[rectangle, fill=purple, line width=0pt] (8.5, -0.2) -- (8.5, 0.2) -- (9.1, 0.2) -- (9.1, -0.2) -- (8.5, -0.2);
		
		
% 		\node at (6.6, 0.65) {$\Delta$};
% 		\node at (8.8, 0.65) {$\Delta$};
% 		\draw[decorate, decoration={brace, amplitude=3pt}, thick] (6.3, 0.28) -- (6.9, 0.28);
		
% 			\draw[decorate, decoration={brace, amplitude=3pt}, thick] (8.5, 0.28) -- (9.1, 0.28);
		
% 		\node[] (deltastar) at (8.1, -1.2) {\large \textcolor{violet}{$\delta^\star$}};
% 		\draw[-, , violet, thick] (8.0, -0.9) -- (8.0, 0.9);

 
% 	\end{tikzpicture}
% 	}
% 	\caption{Illustration of the proof, using $d=4, \delta = \nicefrac{3}{4}$ as an example. In this case, the probability of failure is $\frac{2\Delta}{\varepsilon}.$}
% \end{figure}

\end{proof}


% Let us restate and prove our other theorem.

% \begin{theorem}
%     The Uniform-Min-$\delta$-SR$(\textsc{Linear})$ problem can be approximated over $k^\star$ with additive error $1$ and probability of success $1-\gamma$ in time $\tilde{O}\left( \frac{d^2}{\varepsilon^2\gamma^2}\right)$.
% \end{theorem}
% \begin{proof}[Proof sketch]
%     Sort the features according to \emph{score} again, then we try to find the minimum $k$ such that the top-$k$ most important features give a probability of $\delta$. In particular, we will accept the first $k$ such that the empirical probability is above $\delta + (1-\delta)/2d$. The chances of the ``true'' probability being above $\delta$ will naturally be good, the problem is to guarantee that we won't overshoot by much. The main insight here is that, if $k^*$ is the true optimal answer (thus having probability at least $\delta$), then $k^*+1$ should always have the desired probability; note that the remaining probability is $1-\delta$, and there are $d-{k^*}$ features remaining, so the next most important one must yield at least a gain of 
%     $\frac{1-\delta}{d - k^*} \geq \frac{1 - \delta}{d}$. So by sampling with an error of $\frac{1-\delta}{2d}$ we will recognize this, which by using the fact requires 
% $
%     \tilde{O}\left(\left(\frac{1-\delta}{2d}\right)^2\right)
% $ many samples to ensure with high probability. Note that the problematic case is now when $\delta \approx 1$. 
% \end{proof}
