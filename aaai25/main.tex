%File: anonymous-submission-latex-2025.tex
\documentclass[letterpaper]{article} % DO NOT CHANGE THIS
\usepackage[submission]{aaai25}  % DO NOT CHANGE THIS
\usepackage{times}  % DO NOT CHANGE THIS
\usepackage{helvet}  % DO NOT CHANGE THIS
\usepackage{courier}  % DO NOT CHANGE THIS

\usepackage[hyphens]{url}  % DO NOT CHANGE THIS
\usepackage{graphicx} % DO NOT CHANGE THIS
\urlstyle{rm} % DO NOT CHANGE THIS
\def\UrlFont{\rm}  % DO NOT CHANGE THIS
\usepackage{natbib}  % DO NOT CHANGE THIS AND DO NOT ADD ANY OPTIONS TO IT
\usepackage{caption} % DO NOT CHANGE THIS AND DO NOT ADD ANY OPTIONS TO IT
\frenchspacing  % DO NOT CHANGE THIS
\setlength{\pdfpagewidth}{8.5in} % DO NOT CHANGE THIS
\setlength{\pdfpageheight}{11in} % DO NOT CHANGE THIS
%
% These are recommended to typeset algorithms but not required. See the subsubsection on algorithms. Remove them if you don't have algorithms in your paper.
\usepackage{algorithm}
\usepackage{algorithmic}

%
% These are are recommended to typeset listings but not required. See the subsubsection on listing. Remove this block if you don't have listings in your paper.
\usepackage{newfloat}
\usepackage{listings}
\DeclareCaptionStyle{ruled}{labelfont=normalfont,labelsep=colon,strut=off} % DO NOT CHANGE THIS
\lstset{%
	basicstyle={\footnotesize\ttfamily},% footnotesize acceptable for monospace
	numbers=left,numberstyle=\footnotesize,xleftmargin=2em,% show line numbers, remove this entire line if you don't want the numbers.
	aboveskip=0pt,belowskip=0pt,%
	showstringspaces=false,tabsize=2,breaklines=true}
\floatstyle{ruled}
\newfloat{listing}{tb}{lst}{}
\floatname{listing}{Listing}
%
% Keep the \pdfinfo as shown here. There's no need
% for you to add the /Title and /Author tags.
\pdfinfo{
/TemplateVersion (2025.1)
}

\usepackage{booktabs}  
% DISALLOWED PACKAGES
% \usepackage{authblk} -- This package is specifically forbidden
% \usepackage{balance} -- This package is specifically forbidden
% \usepackage{color (if used in text)
% \usepackage{CJK} -- This package is specifically forbidden
% \usepackage{float} -- This package is specifically forbidden
% \usepackage{flushend} -- This package is specifically forbidden
% \usepackage{fontenc} -- This package is specifically forbidden
% \usepackage{fullpage} -- This package is specifically forbidden
% \usepackage{geometry} -- This package is specifically forbidden
% \usepackage{grffile} -- This package is specifically forbidden
% \usepackage{hyperref} -- This package is specifically forbidden
% \usepackage{navigator} -- This package is specifically forbidden
% (or any other package that embeds links such as navigator or hyperref)
% \indentfirst} -- This package is specifically forbidden
% \layout} -- This package is specifically forbidden
% \multicol} -- This package is specifically forbidden
% \nameref} -- This package is specifically forbidden
% \usepackage{savetrees} -- This package is specifically forbidden
% \usepackage{setspace} -- This package is specifically forbidden
% \usepackage{stfloats} -- This package is specifically forbidden
% \usepackage{tabu} -- This package is specifically forbidden
% \usepackage{titlesec} -- This package is specifically forbidden
% \usepackage{tocbibind} -- This package is specifically forbidden
% \usepackage{ulem} -- This package is specifically forbidden
% \usepackage{wrapfig} -- This package is specifically forbidden
% DISALLOWED COMMANDS
% \nocopyright -- Your paper will not be published if you use this command
% \addtolength -- This command may not be used
% \balance -- This command may not be used
% \baselinestretch -- Your paper will not be published if you use this command
% \clearpage -- No page breaks of any kind may be used for the final version of your paper
% \columnsep -- This command may not be used
% \newpage -- No page breaks of any kind may be used for the final version of your paper
% \pagebreak -- No page breaks of any kind may be used for the final version of your paperr
% \pagestyle -- This command may not be used
% \tiny -- This is not an acceptable font size.
% \vspace{- -- No negative value may be used in proximity of a caption, figure, table, section, subsection, subsubsection, or reference
% \vskip{- -- No negative value may be used to alter spacing above or below a caption, figure, table, section, subsection, subsubsection, or reference

\setcounter{secnumdepth}{0} %May be changed to 1 or 2 if section numbers are desired.

% The file aaai25.sty is the style file for AAAI Press
% proceedings, working notes, and technical reports.
%

% Title

% Your title must be in mixed case, not sentence case.
% That means all verbs (including short verbs like be, is, using,and go),
% nouns, adverbs, adjectives should be capitalized, including both words in hyphenated terms, while
% articles, conjunctions, and prepositions are lower case unless they
% directly follow a colon or long dash
\title{Probabilistic Explanations for Linear Models}
\author{
    %Authors
    % All authors must be in the same font size and format.
    Written by AAAI Press Staff\textsuperscript{\rm 1}\thanks{With help from the AAAI Publications Committee.}\\
    AAAI Style Contributions by Pater Patel Schneider,
    Sunil Issar,\\
    J. Scott Penberthy,
    George Ferguson,
    Hans Guesgen,
    Francisco Cruz\equalcontrib,
    Marc Pujol-Gonzalez\equalcontrib
}
\affiliations{
    %Afiliations
    \textsuperscript{\rm 1}Association for the Advancement of Artificial Intelligence\\
    % If you have multiple authors and multiple affiliations
    % use superscripts in text and roman font to identify them.
    % For example,

    % Sunil Issar\textsuperscript{\rm 2},
    % J. Scott Penberthy\textsuperscript{\rm 3},
    % George Ferguson\textsuperscript{\rm 4},
    % Hans Guesgen\textsuperscript{\rm 5}
    % Note that the comma should be placed after the superscript

    1101 Pennsylvania Ave, NW Suite 300\\
    Washington, DC 20004 USA\\
    % email address must be in roman text type, not monospace or sans serif
    proceedings-questions@aaai.org
%
% See more examples next
}

%Example, Single Author, ->> remove \iffalse,\fi and place them surrounding AAAI title to use it
\iffalse
\title{My Publication Title --- Single Author}
\author {
    Author Name
}
\affiliations{
    Affiliation\\
    Affiliation Line 2\\
    name@example.com
}
\fi

\iffalse
%Example, Multiple Authors, ->> remove \iffalse,\fi and place them surrounding AAAI title to use it
\title{My Publication Title --- Multiple Authors}
\author {
    % Authors
    First Author Name\textsuperscript{\rm 1},
    Second Author Name\textsuperscript{\rm 2},
    Third Author Name\textsuperscript{\rm 1}
}
\affiliations {
    % Affiliations
    \textsuperscript{\rm 1}Affiliation 1\\
    \textsuperscript{\rm 2}Affiliation 2\\
    firstAuthor@affiliation1.com, secondAuthor@affilation2.com, thirdAuthor@affiliation1.com
}
\fi




%%%%% NEW MATH DEFINITIONS %%%%%

\usepackage{amsmath,amsfonts,bm}

% Mark sections of captions for referring to divisions of figures
\newcommand{\figleft}{{\em (Left)}}
\newcommand{\figcenter}{{\em (Center)}}
\newcommand{\figright}{{\em (Right)}}
\newcommand{\figtop}{{\em (Top)}}
\newcommand{\figbottom}{{\em (Bottom)}}
\newcommand{\captiona}{{\em (a)}}
\newcommand{\captionb}{{\em (b)}}
\newcommand{\captionc}{{\em (c)}}
\newcommand{\captiond}{{\em (d)}}

% Highlight a newly defined term
\newcommand{\newterm}[1]{{\bf #1}}


% Figure reference, lower-case.
\def\figref#1{figure~\ref{#1}}
% Figure reference, capital. For start of sentence
\def\Figref#1{Figure~\ref{#1}}
\def\twofigref#1#2{figures \ref{#1} and \ref{#2}}
\def\quadfigref#1#2#3#4{figures \ref{#1}, \ref{#2}, \ref{#3} and \ref{#4}}
% Section reference, lower-case.
\def\secref#1{section~\ref{#1}}
% Section reference, capital.
\def\Secref#1{Section~\ref{#1}}
% Reference to two sections.
\def\twosecrefs#1#2{sections \ref{#1} and \ref{#2}}
% Reference to three sections.
\def\secrefs#1#2#3{sections \ref{#1}, \ref{#2} and \ref{#3}}
% Reference to an equation, lower-case.
\def\eqref#1{equation~\ref{#1}}
% Reference to an equation, upper case
\def\Eqref#1{Equation~\ref{#1}}
% A raw reference to an equation---avoid using if possible
\def\plaineqref#1{\ref{#1}}
% Reference to a chapter, lower-case.
\def\chapref#1{chapter~\ref{#1}}
% Reference to an equation, upper case.
\def\Chapref#1{Chapter~\ref{#1}}
% Reference to a range of chapters
\def\rangechapref#1#2{chapters\ref{#1}--\ref{#2}}
% Reference to an algorithm, lower-case.
\def\algref#1{algorithm~\ref{#1}}
% Reference to an algorithm, upper case.
\def\Algref#1{Algorithm~\ref{#1}}
\def\twoalgref#1#2{algorithms \ref{#1} and \ref{#2}}
\def\Twoalgref#1#2{Algorithms \ref{#1} and \ref{#2}}
% Reference to a part, lower case
\def\partref#1{part~\ref{#1}}
% Reference to a part, upper case
\def\Partref#1{Part~\ref{#1}}
\def\twopartref#1#2{parts \ref{#1} and \ref{#2}}

\def\ceil#1{\lceil #1 \rceil}
\def\floor#1{\lfloor #1 \rfloor}
\def\1{\bm{1}}
\newcommand{\train}{\mathcal{D}}
\newcommand{\valid}{\mathcal{D_{\mathrm{valid}}}}
\newcommand{\test}{\mathcal{D_{\mathrm{test}}}}

\def\eps{{\epsilon}}


% Random variables
\def\reta{{\textnormal{$\eta$}}}
\def\ra{{\textnormal{a}}}
\def\rb{{\textnormal{b}}}
\def\rc{{\textnormal{c}}}
\def\rd{{\textnormal{d}}}
\def\re{{\textnormal{e}}}
\def\rf{{\textnormal{f}}}
\def\rg{{\textnormal{g}}}
\def\rh{{\textnormal{h}}}
\def\ri{{\textnormal{i}}}
\def\rj{{\textnormal{j}}}
\def\rk{{\textnormal{k}}}
\def\rl{{\textnormal{l}}}
% rm is already a command, just don't name any random variables m
\def\rn{{\textnormal{n}}}
\def\ro{{\textnormal{o}}}
\def\rp{{\textnormal{p}}}
\def\rq{{\textnormal{q}}}
\def\rr{{\textnormal{r}}}
\def\rs{{\textnormal{s}}}
\def\rt{{\textnormal{t}}}
\def\ru{{\textnormal{u}}}
\def\rv{{\textnormal{v}}}
\def\rw{{\textnormal{w}}}
\def\rx{{\textnormal{x}}}
\def\ry{{\textnormal{y}}}
\def\rz{{\textnormal{z}}}

% Random vectors
\def\rvepsilon{{\mathbf{\epsilon}}}
\def\rvtheta{{\mathbf{\theta}}}
\def\rva{{\mathbf{a}}}
\def\rvb{{\mathbf{b}}}
\def\rvc{{\mathbf{c}}}
\def\rvd{{\mathbf{d}}}
\def\rve{{\mathbf{e}}}
\def\rvf{{\mathbf{f}}}
\def\rvg{{\mathbf{g}}}
\def\rvh{{\mathbf{h}}}
\def\rvu{{\mathbf{i}}}
\def\rvj{{\mathbf{j}}}
\def\rvk{{\mathbf{k}}}
\def\rvl{{\mathbf{l}}}
\def\rvm{{\mathbf{m}}}
\def\rvn{{\mathbf{n}}}
\def\rvo{{\mathbf{o}}}
\def\rvp{{\mathbf{p}}}
\def\rvq{{\mathbf{q}}}
\def\rvr{{\mathbf{r}}}
\def\rvs{{\mathbf{s}}}
\def\rvt{{\mathbf{t}}}
\def\rvu{{\mathbf{u}}}
\def\rvv{{\mathbf{v}}}
\def\rvw{{\mathbf{w}}}
\def\rvx{{\mathbf{x}}}
\def\rvy{{\mathbf{y}}}
\def\rvz{{\mathbf{z}}}

% Elements of random vectors
\def\erva{{\textnormal{a}}}
\def\ervb{{\textnormal{b}}}
\def\ervc{{\textnormal{c}}}
\def\ervd{{\textnormal{d}}}
\def\erve{{\textnormal{e}}}
\def\ervf{{\textnormal{f}}}
\def\ervg{{\textnormal{g}}}
\def\ervh{{\textnormal{h}}}
\def\ervi{{\textnormal{i}}}
\def\ervj{{\textnormal{j}}}
\def\ervk{{\textnormal{k}}}
\def\ervl{{\textnormal{l}}}
\def\ervm{{\textnormal{m}}}
\def\ervn{{\textnormal{n}}}
\def\ervo{{\textnormal{o}}}
\def\ervp{{\textnormal{p}}}
\def\ervq{{\textnormal{q}}}
\def\ervr{{\textnormal{r}}}
\def\ervs{{\textnormal{s}}}
\def\ervt{{\textnormal{t}}}
\def\ervu{{\textnormal{u}}}
\def\ervv{{\textnormal{v}}}
\def\ervw{{\textnormal{w}}}
\def\ervx{{\textnormal{x}}}
\def\ervy{{\textnormal{y}}}
\def\ervz{{\textnormal{z}}}

% Random matrices
\def\rmA{{\mathbf{A}}}
\def\rmB{{\mathbf{B}}}
\def\rmC{{\mathbf{C}}}
\def\rmD{{\mathbf{D}}}
\def\rmE{{\mathbf{E}}}
\def\rmF{{\mathbf{F}}}
\def\rmG{{\mathbf{G}}}
\def\rmH{{\mathbf{H}}}
\def\rmI{{\mathbf{I}}}
\def\rmJ{{\mathbf{J}}}
\def\rmK{{\mathbf{K}}}
\def\rmL{{\mathbf{L}}}
\def\rmM{{\mathbf{M}}}
\def\rmN{{\mathbf{N}}}
\def\rmO{{\mathbf{O}}}
\def\rmP{{\mathbf{P}}}
\def\rmQ{{\mathbf{Q}}}
\def\rmR{{\mathbf{R}}}
\def\rmS{{\mathbf{S}}}
\def\rmT{{\mathbf{T}}}
\def\rmU{{\mathbf{U}}}
\def\rmV{{\mathbf{V}}}
\def\rmW{{\mathbf{W}}}
\def\rmX{{\mathbf{X}}}
\def\rmY{{\mathbf{Y}}}
\def\rmZ{{\mathbf{Z}}}

% Elements of random matrices
\def\ermA{{\textnormal{A}}}
\def\ermB{{\textnormal{B}}}
\def\ermC{{\textnormal{C}}}
\def\ermD{{\textnormal{D}}}
\def\ermE{{\textnormal{E}}}
\def\ermF{{\textnormal{F}}}
\def\ermG{{\textnormal{G}}}
\def\ermH{{\textnormal{H}}}
\def\ermI{{\textnormal{I}}}
\def\ermJ{{\textnormal{J}}}
\def\ermK{{\textnormal{K}}}
\def\ermL{{\textnormal{L}}}
\def\ermM{{\textnormal{M}}}
\def\ermN{{\textnormal{N}}}
\def\ermO{{\textnormal{O}}}
\def\ermP{{\textnormal{P}}}
\def\ermQ{{\textnormal{Q}}}
\def\ermR{{\textnormal{R}}}
\def\ermS{{\textnormal{S}}}
\def\ermT{{\textnormal{T}}}
\def\ermU{{\textnormal{U}}}
\def\ermV{{\textnormal{V}}}
\def\ermW{{\textnormal{W}}}
\def\ermX{{\textnormal{X}}}
\def\ermY{{\textnormal{Y}}}
\def\ermZ{{\textnormal{Z}}}

% Vectors
\def\vzero{{\bm{0}}}
\def\vone{{\bm{1}}}
\def\vmu{{\bm{\mu}}}
\def\vtheta{{\bm{\theta}}}
\def\va{{\bm{a}}}
\def\vb{{\bm{b}}}
\def\vc{{\bm{c}}}
\def\vd{{\bm{d}}}
\def\ve{{\bm{e}}}
\def\vf{{\bm{f}}}
\def\vg{{\bm{g}}}
\def\vh{{\bm{h}}}
\def\vi{{\bm{i}}}
\def\vj{{\bm{j}}}
\def\vk{{\bm{k}}}
\def\vl{{\bm{l}}}
\def\vm{{\bm{m}}}
\def\vn{{\bm{n}}}
\def\vo{{\bm{o}}}
\def\vp{{\bm{p}}}
\def\vq{{\bm{q}}}
\def\vr{{\bm{r}}}
\def\vs{{\bm{s}}}
\def\vt{{\bm{t}}}
\def\vu{{\bm{u}}}
\def\vv{{\bm{v}}}
\def\vw{{\bm{w}}}
\def\vx{{\bm{x}}}
\def\vy{{\bm{y}}}
\def\vz{{\bm{z}}}

% Elements of vectors
\def\evalpha{{\alpha}}
\def\evbeta{{\beta}}
\def\evepsilon{{\epsilon}}
\def\evlambda{{\lambda}}
\def\evomega{{\omega}}
\def\evmu{{\mu}}
\def\evpsi{{\psi}}
\def\evsigma{{\sigma}}
\def\evtheta{{\theta}}
\def\eva{{a}}
\def\evb{{b}}
\def\evc{{c}}
\def\evd{{d}}
\def\eve{{e}}
\def\evf{{f}}
\def\evg{{g}}
\def\evh{{h}}
\def\evi{{i}}
\def\evj{{j}}
\def\evk{{k}}
\def\evl{{l}}
\def\evm{{m}}
\def\evn{{n}}
\def\evo{{o}}
\def\evp{{p}}
\def\evq{{q}}
\def\evr{{r}}
\def\evs{{s}}
\def\evt{{t}}
\def\evu{{u}}
\def\evv{{v}}
\def\evw{{w}}
\def\evx{{x}}
\def\evy{{y}}
\def\evz{{z}}

% Matrix
\def\mA{{\bm{A}}}
\def\mB{{\bm{B}}}
\def\mC{{\bm{C}}}
\def\mD{{\bm{D}}}
\def\mE{{\bm{E}}}
\def\mF{{\bm{F}}}
\def\mG{{\bm{G}}}
\def\mH{{\bm{H}}}
\def\mI{{\bm{I}}}
\def\mJ{{\bm{J}}}
\def\mK{{\bm{K}}}
\def\mL{{\bm{L}}}
\def\mM{{\bm{M}}}
\def\mN{{\bm{N}}}
\def\mO{{\bm{O}}}
\def\mP{{\bm{P}}}
\def\mQ{{\bm{Q}}}
\def\mR{{\bm{R}}}
\def\mS{{\bm{S}}}
\def\mT{{\bm{T}}}
\def\mU{{\bm{U}}}
\def\mV{{\bm{V}}}
\def\mW{{\bm{W}}}
\def\mX{{\bm{X}}}
\def\mY{{\bm{Y}}}
\def\mZ{{\bm{Z}}}
\def\mBeta{{\bm{\beta}}}
\def\mPhi{{\bm{\Phi}}}
\def\mLambda{{\bm{\Lambda}}}
\def\mSigma{{\bm{\Sigma}}}

% Tensor
\DeclareMathAlphabet{\mathsfit}{\encodingdefault}{\sfdefault}{m}{sl}
\SetMathAlphabet{\mathsfit}{bold}{\encodingdefault}{\sfdefault}{bx}{n}
\newcommand{\tens}[1]{\bm{\mathsfit{#1}}}
\def\tA{{\tens{A}}}
\def\tB{{\tens{B}}}
\def\tC{{\tens{C}}}
\def\tD{{\tens{D}}}
\def\tE{{\tens{E}}}
\def\tF{{\tens{F}}}
\def\tG{{\tens{G}}}
\def\tH{{\tens{H}}}
\def\tI{{\tens{I}}}
\def\tJ{{\tens{J}}}
\def\tK{{\tens{K}}}
\def\tL{{\tens{L}}}
\def\tM{{\tens{M}}}
\def\tN{{\tens{N}}}
\def\tO{{\tens{O}}}
\def\tP{{\tens{P}}}
\def\tQ{{\tens{Q}}}
\def\tR{{\tens{R}}}
\def\tS{{\tens{S}}}
\def\tT{{\tens{T}}}
\def\tU{{\tens{U}}}
\def\tV{{\tens{V}}}
\def\tW{{\tens{W}}}
\def\tX{{\tens{X}}}
\def\tY{{\tens{Y}}}
\def\tZ{{\tens{Z}}}


% Graph
\def\gA{{\mathcal{A}}}
\def\gB{{\mathcal{B}}}
\def\gC{{\mathcal{C}}}
\def\gD{{\mathcal{D}}}
\def\gE{{\mathcal{E}}}
\def\gF{{\mathcal{F}}}
\def\gG{{\mathcal{G}}}
\def\gH{{\mathcal{H}}}
\def\gI{{\mathcal{I}}}
\def\gJ{{\mathcal{J}}}
\def\gK{{\mathcal{K}}}
\def\gL{{\mathcal{L}}}
\def\gM{{\mathcal{M}}}
\def\gN{{\mathcal{N}}}
\def\gO{{\mathcal{O}}}
\def\gP{{\mathcal{P}}}
\def\gQ{{\mathcal{Q}}}
\def\gR{{\mathcal{R}}}
\def\gS{{\mathcal{S}}}
\def\gT{{\mathcal{T}}}
\def\gU{{\mathcal{U}}}
\def\gV{{\mathcal{V}}}
\def\gW{{\mathcal{W}}}
\def\gX{{\mathcal{X}}}
\def\gY{{\mathcal{Y}}}
\def\gZ{{\mathcal{Z}}}

% Sets
\def\sA{{\mathbb{A}}}
\def\sB{{\mathbb{B}}}
\def\sC{{\mathbb{C}}}
\def\sD{{\mathbb{D}}}
% Don't use a set called E, because this would be the same as our symbol
% for expectation.
\def\sF{{\mathbb{F}}}
\def\sG{{\mathbb{G}}}
\def\sH{{\mathbb{H}}}
\def\sI{{\mathbb{I}}}
\def\sJ{{\mathbb{J}}}
\def\sK{{\mathbb{K}}}
\def\sL{{\mathbb{L}}}
\def\sM{{\mathbb{M}}}
\def\sN{{\mathbb{N}}}
\def\sO{{\mathbb{O}}}
\def\sP{{\mathbb{P}}}
\def\sQ{{\mathbb{Q}}}
\def\sR{{\mathbb{R}}}
\def\sS{{\mathbb{S}}}
\def\sT{{\mathbb{T}}}
\def\sU{{\mathbb{U}}}
\def\sV{{\mathbb{V}}}
\def\sW{{\mathbb{W}}}
\def\sX{{\mathbb{X}}}
\def\sY{{\mathbb{Y}}}
\def\sZ{{\mathbb{Z}}}

% Entries of a matrix
\def\emLambda{{\Lambda}}
\def\emA{{A}}
\def\emB{{B}}
\def\emC{{C}}
\def\emD{{D}}
\def\emE{{E}}
\def\emF{{F}}
\def\emG{{G}}
\def\emH{{H}}
\def\emI{{I}}
\def\emJ{{J}}
\def\emK{{K}}
\def\emL{{L}}
\def\emM{{M}}
\def\emN{{N}}
\def\emO{{O}}
\def\emP{{P}}
\def\emQ{{Q}}
\def\emR{{R}}
\def\emS{{S}}
\def\emT{{T}}
\def\emU{{U}}
\def\emV{{V}}
\def\emW{{W}}
\def\emX{{X}}
\def\emY{{Y}}
\def\emZ{{Z}}
\def\emSigma{{\Sigma}}

% entries of a tensor
% Same font as tensor, without \bm wrapper
\newcommand{\etens}[1]{\mathsfit{#1}}
\def\etLambda{{\etens{\Lambda}}}
\def\etA{{\etens{A}}}
\def\etB{{\etens{B}}}
\def\etC{{\etens{C}}}
\def\etD{{\etens{D}}}
\def\etE{{\etens{E}}}
\def\etF{{\etens{F}}}
\def\etG{{\etens{G}}}
\def\etH{{\etens{H}}}
\def\etI{{\etens{I}}}
\def\etJ{{\etens{J}}}
\def\etK{{\etens{K}}}
\def\etL{{\etens{L}}}
\def\etM{{\etens{M}}}
\def\etN{{\etens{N}}}
\def\etO{{\etens{O}}}
\def\etP{{\etens{P}}}
\def\etQ{{\etens{Q}}}
\def\etR{{\etens{R}}}
\def\etS{{\etens{S}}}
\def\etT{{\etens{T}}}
\def\etU{{\etens{U}}}
\def\etV{{\etens{V}}}
\def\etW{{\etens{W}}}
\def\etX{{\etens{X}}}
\def\etY{{\etens{Y}}}
\def\etZ{{\etens{Z}}}

% The true underlying data generating distribution
\newcommand{\pdata}{p_{\rm{data}}}
% The empirical distribution defined by the training set
\newcommand{\ptrain}{\hat{p}_{\rm{data}}}
\newcommand{\Ptrain}{\hat{P}_{\rm{data}}}
% The model distribution
\newcommand{\pmodel}{p_{\rm{model}}}
\newcommand{\Pmodel}{P_{\rm{model}}}
\newcommand{\ptildemodel}{\tilde{p}_{\rm{model}}}
% Stochastic autoencoder distributions
\newcommand{\pencode}{p_{\rm{encoder}}}
\newcommand{\pdecode}{p_{\rm{decoder}}}
\newcommand{\precons}{p_{\rm{reconstruct}}}

\newcommand{\laplace}{\mathrm{Laplace}} % Laplace distribution

\newcommand{\E}{\mathbb{E}}
\newcommand{\Ls}{\mathcal{L}}
\newcommand{\R}{\mathbb{R}}
\newcommand{\emp}{\tilde{p}}
\newcommand{\lr}{\alpha}
\newcommand{\reg}{\lambda}
\newcommand{\rect}{\mathrm{rectifier}}
\newcommand{\softmax}{\mathrm{softmax}}
\newcommand{\sigmoid}{\sigma}
\newcommand{\softplus}{\zeta}
\newcommand{\KL}{D_{\mathrm{KL}}}
\newcommand{\Var}{\mathrm{Var}}
\newcommand{\standarderror}{\mathrm{SE}}
\newcommand{\Cov}{\mathrm{Cov}}
% Wolfram Mathworld says $L^2$ is for function spaces and $\ell^2$ is for vectors
% But then they seem to use $L^2$ for vectors throughout the site, and so does
% wikipedia.
\newcommand{\normlzero}{L^0}
\newcommand{\normlone}{L^1}
\newcommand{\normltwo}{L^2}
\newcommand{\normlp}{L^p}
\newcommand{\normmax}{L^\infty}

\newcommand{\parents}{Pa} % See usage in notation.tex. Chosen to match Daphne's book.

\DeclareMathOperator*{\argmax}{arg\,max}
\DeclareMathOperator*{\argmin}{arg\,min}

\DeclareMathOperator{\sign}{sign}
\DeclareMathOperator{\Tr}{Tr}
\let\ab\allowbreak


\usepackage{nicefrac}
\usepackage{todonotes}
\usepackage{amsthm, amsmath, amssymb}
\usepackage[capitalise,nameinlink]{cleveref}
\crefname{lemma}{Lemma}{Lemmas}
\crefname{fact}{Fact}{Facts}

\newtheorem{fact}{Fact}
\newtheorem{theorem}{Theorem}
\newtheorem{lemma}{Lemma}
\newtheorem{definition}{Definition}
\newtheorem{proposition}{Proposition}
\newtheorem{example}{Example}
\newtheorem{open}{Open Problem}

\newcommand{\M}{\mathcal{M}}
\newcommand{\D}{\mathbf{D}}
\newcommand{\Lin}{\mathcal{L}}
\newcommand{\ptime}{\mathrm{P}}
\newcommand{\Knapsack}{\textsc{Knapsack}}

\DeclareMathOperator{\comp}{Comp}

\newcommand{\csproblem}[3]{
    \begin{center}
    \fbox{\begin{tabular}{lp{5.5cm}}
    {\small PROBLEM:} : & #1 \\
    {\small INPUT} : & #2 %of dimension $d$,                                                                                
    %\\ & $x \in \{0,1\}^d$ an instance, and $0 \leq \delta \leq 1$                                                         
    \\
    {\small OUTPUT} : & #3\\
    \end{tabular}}
    \end{center}
    }

\begin{document}

\maketitle

\begin{abstract}
Formal XAI is an emerging field that focuses on providing explanations
with mathematical guarantees for the decisions made by machine
learning models. Recent work in this area has centered on the
computation of ``sufficient reasons''. In this context, given a model $\M$
and an instance $\vx$, a sufficient reason for the decision $\M(\vx)$ is a
subset $S$ of the features of $\vx$ such that for any other instance $\vz$
compatible with $S$, it holds that $\M(\vx) = \M(\vz)$. Intuitively, this means
that the features in $S$ are sufficient to fully justify the
classification of $\vx$ by $\M$.
For sufficient reasons to be useful in practice, they should be as
small as possible. A natural way to reduce the size of sufficient
reasons, that has received recent attention, is to consider a
probabilistic relaxation; the probability of $\M(\vx) = \M(\vz)$ must
be at least some value $\delta \in (0,1]$, where $\vz$ is a random
  instance compatible with $\vx$. In particular, $\delta$ is a
  parameter provided by the user, that represents a trade-off between
  the explanation quality and its size. Naturally, one would aim to
  compute $\delta$-sufficient reasons that are minimal in size. In this
  paper, we study the problem of computing such minimum
  $\delta$-sufficient reasons for linear models. On the negative side, we
  prove that this problem is computationally hard if one aims for
  exact solutions. On the positive side, we consider natural forms of
  approximation for this problem and show that they can be achieved in
  polynomial time.
\end{abstract}


\section{Introduction}

Explaining the decisions of Machine Learning classifiers is a fundamental problem in XAI (Explainable AI), and doing so with formal mathematical guarantees on the quality, size, and semantics of the explanations is in turn the core of \emph{Formal XAI}~\cite{formal-xai}. 
Within formal XAI, one of the most studied kinds of explanations is that of \emph{sufficient reasons}~\cite{Darwiche_Hirth_2020}, which aim to explain a decision $\M(\vx) = 1$ by presenting a subset $S$ of the features of the input $\vx$ that implies $\M(\vz) = 1$  for any $\vz$ that agrees with $\vx$ on $S$. 
In the language of theoretical computer science, these correspond to \emph{certificates} for $\M(\vx)$.

\begin{example}
Consider a binary classifier~$\M$ defined as 
	\[
	\M(\vx) = \left(x_1 \lor \overline{x_3}\right) \land   \left(x_2 \lor \overline{x_1}\right) \land \left(x_4 \lor x_3\right),
	\]
	and the input instance $\vx = \left( 1, \,  1, \, 0, \, 1 \right)$. We can say that $\M(\vx)$ ``because'' $x_1 = 1, x_2 = 1$, and $x_4 = 1$, as they are sufficient to determine the value of $\M(\vx)$ regardless of $x_3$.
	\label{ex:sufficient-reason}
\end{example}

Let us start formalizing the framework for our work.  First, we consider binary boolean models $\M\colon \{0, 1\}^d \to \{0, 1\}$. Despite our domain being binary, we will need a third value, $\bot$, to denote \emph{``unknown''} values.  For example, we may represent a person who \emph{does} have a car, \emph{does not} have a house, and for whom we do not know if they have a pet or not, as $\left(1, \, 0, \, \bot\right)$. 
We say elements of $\{0, 1, \bot\}^d$ are \emph{partial instances}, while elements of $\{0, 1\}^d$ are simply \emph{instances}. To illustrate, in~\Cref{ex:sufficient-reason} we used the partial instance $\vy = \left(1, \, 1, \, \bot, \, 1 \right)$ to explain $\M(\vx) = 1$.
We use the notation $\vy \subseteq \vx$ to denote that the (partial) instance $\vx$ \emph{``fills in''} values of the partial instance $\vy$; more formally, we use $\vy \subseteq \vx$ to mean that $y_i = \bot \lor y_i = x_i$ for every $i \in [d]$. Finally, for any partial instance $\vy$ we denote by $\comp(\vy)$ the set of instances $\vx$ such that $\vy \subseteq \vx$, thinking of $\comp(\vy)$ as the set of \emph{completions} of $\vy$. One can define sufficient reasons as follows with this notation.

\begin{definition}[Sufficient Reason~\citep{Darwiche_Hirth_2020}]
	We say $\vy$ is a \emph{sufficient reason} for $\vx$ if for any completion $\vz \in \comp(\vy)$ it holds that $\M(\vx) = \M(\vz)$.
	\label{def:sufficient-reason}
\end{definition}
A crucial factor for the helpfulness of sufficient reasons as explanations is their size; even though $\vx$ is always a sufficient reason for its own classification, we long for explanations that are much smaller than $\vx$ itself. \citet{millerMagicalNumberSeven1956}, for instance, goes on to say that explanations consisting of more than $9$ features are probably too large for human stakeholders. In general, empirical research suggests that explanations ought to be small~\cite{Narayanan_Chen_He_Kim_Gershman_Doshi-Velez_2018, Lage_Chen_He_Narayanan_Kim_Gershman_Doshi-Velez_2019}.
There are several ways of formalizing the succinctness we desire for sufficient reasons:

\begin{itemize}
    \item \textbf{(Minimum Size)} For a sufficient reason $\vy$, we define its \emph{explanation size} $|\vy|_e$ as the number of defined features in $\vy$, or equivalently, $|\vy|_e := d - |\vy|_\bot$, where $|\vy|_\bot$ is the number of features of $\vy$ taking $\bot$. See e.g.,~\cite{NEURIPS2020_b1adda14}.\footnote{When talking about a partial instance $\vy$, we will use the ``size'' of $\vy$ to mean $|\vy|_e$.}
    \item \textbf{(Minimality)} We say a sufficient reason $\vy$ for a pair $(\M, \vx)$ is \emph{minimal} if there is no other sufficient reason $\vy'$ for $(\M, \vx)$ such that $\vy' \subsetneq \vy$. In fact, the original definition of sufficient reasons of~\citet{Darwiche_Hirth_2020} includes minimality as a requirement, and so is the case under the \emph{``abductive explanation''} naming~\cite{Ignatiev_Narodytska_Asher_Marques-Silva_2021}.
    \item \textbf{(Relative to average explanation)} \citet{blanc2021provably} compute explanations that are small relative to the \emph{``certificate complexity''} of the classifier $\M$, meaning the average size of the minimum sufficient reason where the average is taken over all possible instances $\vx$.
\end{itemize}

Nevertheless, there is a path toward even smaller explanations: \emph{probabilistic} sufficient reasons~\cite{Waldchen_MacDonald_Hauch_Kutyniok_2021, Izza_Huang_Ignatiev_Narodytska_Cooper_Marques-Silva_2023}. 
As will be shown in Example 2., and is noted as a remark by~\citet{blanc2021provably}, these can be arbitrarily smaller than minimum size sufficient reasons.

The main idea of probabilistic sufficient reasons is to relax the condition \emph{``all completions of the explanation $\vy$ have the same class as $\vx$''} to \emph{``a random completion of $\vy$ has the same class as $\vx$ with high probability''}. To define this formally let us start by assuming an instance distribution $\D$, meaning that an input $\vx$ has probability $\D(\vx)$ of arising. For example, if one were to deploy an algorithm for handwritten-digit classification over $28 \times 28$ binary images (as in a binarized version of MNIST~\cite{deng2012mnist}), then not all $2^{784}$ inputs would be equally likely to appear; those that \emph{look like digits} would be much more likely.
When considering a partial instance $\vy$, we define $\D_{\vy}$ as the distribution $\D$ conditioned on being a completion of $\vy$, which formally means that 
\[
	\D_{\vy}(\vz) = \frac{\D(\vz)}{\sum_{\vw \in \comp(\vy)} \D(\vw)},
\]
for any $\vz \in \comp(\vy)$, and $\D_{\vy}(\vv) = 0$ if $\vv \not\in \comp(\vy)$.
 
With this notation we can define $\delta$-sufficient reasons, also known as $\delta$-relevant sets~\cite{Izza2021EfficientEW}.
\begin{definition}[\cite{Waldchen_MacDonald_Hauch_Kutyniok_2021}]
	For any $\delta \in [0, 1]$, a $\delta$-sufficient reason ($\delta$-SR) for an instance $\vx$, is a partial instance $\vy \subseteq \vx$ such that
	\[
		\Pr_{\vz \sim \D_{\vy}}\big[\M(\vz) = \M(\vx)\big] \geq \delta.
	\]
	\label{def:delta-SR}
\end{definition}
%
Note immediately that~\Cref{def:delta-SR} and~\Cref{def:sufficient-reason} coincide when $\delta = 1$.
\subsection{The size of $\delta$-SRs}

Interestingly, even a $0.999999$-SR can be arbitrarily smaller, in terms of defined features, than the smallest sufficient reason (i.e., $1$-SR) for a pair $(\M, \vx)$, even when $\M$ is a linear model, as we will illustrate in~\Cref{ex:delta-sr-size}. Before providing the example, let us define linear models.
%  and a concentration bound that will be used throughout the paper.

\begin{definition}
	A (binary) linear model $\Lin$ of dimension $d$ is a pair $(\vw, t)$, where $\vw \in \mathbb{Q}^d$ and $t \in \mathbb{Q}$. Its classification over an instance $\vx$ is defined simply as 
	\[
		\Lin(\vx) = \begin{cases}
			1 & \text{if } \vx \cdot \vw \geq t\\
			0 & \text{otherwise}.
		\end{cases}
	\]
	\label{def:linear-models}
\end{definition}

% \begin{lemma}[Chernoff-Hoeffding bound]
% Let $X$ be a finite sum of independent Bernoulli variables, with $\E[X] = \mu > 0$. Then, for any $t \geq 0$, we have
% \[
% \Pr \Big[ \left|X - \mu\right| \geq t \Big] \leq 2\exp\left(\frac{-t^2}{3 \mu} \right).
% \] 

% \label{lemma:chernoff}	
% \end{lemma}

\begin{example}
Consider a linear model $L$ of dimension $d= 1000$ with parameters $t = 1250$ and
$$
	\vw = (
		1000, \, 1, \, 1, \,  1, \,  1, \,  \ldots, \, 1
	).
$$
Let the instance $\vx$ be 
	$(
		1, \, 1, \, 1, \,  1, \,  1, \,  \ldots, \, 1
	),$ so that clearly $\Lin(\vx) = 1$.
One can easily see that any $1$-SR for $\vx$ under $\Lin$ has size $251$, as it must include the first feature and any $250$ other features.
However, if we consider $\D$ to be a uniform distribution, and $\vy = (
		1, \, \bot, \, \bot, \,  \bot,  \ldots, \, \bot
	)$, then a simple application of the Chernoff-Hoeffding concentration bound gives that
\[
	\Pr_{\vz \sim \D(\vy)}\Big[ \Lin(\vz) = 1\Big] \geq  0.999999.\]
	This suggests that we might say $\Lin(\vx) = 1$ ``because'' $x_1 = 1$; formally, $\vy$ is a $0.999999$-SR, and $251$ times smaller than any $1$-SR for $\Lin(\vx)$.
\label{ex:delta-sr-size}
\end{example}

In general, if we let $k^\star(\M, \vx, \delta)$ be the size of the smallest $\delta$-SR for $(\M, \vx)$ under the uniform distribution, we can prove the following simple proposition.

\begin{proposition}
	Even when restricted to the class of linear models, for any $\delta \in (0, 1]$, and any $\varepsilon > 0$, there are pairs $(\M, \vx)$  such that
	\[ 
		\frac{k^\star(\M, \vx, \delta)}{k^\star(\M, \vx, \delta-\varepsilon)} = \Omega(d).
	\]
\end{proposition}


\citet{Waldchen_MacDonald_Hauch_Kutyniok_2021} showed that computing the smallest $\delta$-SR for arbitrary classifiers is hard for $\textrm{NP}^{\textrm{PP}}$, and that no algorithm can achieve an approximation factor (in terms of $k^\star$) of $d^{1-\alpha}$, where $d$ is the dimension of the classifier and $\alpha > 0$, unless $\ptime = \textrm{NP}$. Aftewards,~\citet{Arenas_Barcelo_Romero_Subercaseaux_2022} showed that even for the restricted class of decision trees, usually considered the interpretable, smallest $\delta$-SRs cannot be computed unless $\ptime = \textrm{NP}$, and furthermore,~\citet{Kozachinskiy_2023} proved that the approximation task is also hard for decision trees. Notably, none of these hardness results rely on the distribution $\D$ being complicated, as they were proved taking $\D = \textrm{Uniform}(\{0, 1\}^d)$. 
In stark contrast, we will be able to provide positive approximability results in terms of $\delta$ for linear models, which is, to the best of our knowledge, the first positive theoretical result in the area.
Let us formally define the computational problem at hand, assuming the uniform distribution over $\{0, 1\}^d$, which we will simply denote by $\mathcal{U}$ from now on.



 \csproblem{Uniform-Min-$\delta$-SR$(\mathcal{C})$}{a model $\M \in \mathcal{C}$, an instance $\vx$, a value $\delta \in [0, 1]$, and an integer $k \geq 0$.}{\textsc{\bf Yes} if $k^\star(\M, \vx, \delta) \leq k$, and \textsc{\bf No} otherwise.}
 
Unfortunately, it turns out that even Uniform-Min-$\delta$-SR$(\textsc{Linear})$ is hard, as the amount
\[
    \Pr_{\vz \sim  \mathcal{U}_\vy}\Big[\Lin(\vz) = 1\Big]
\]
cannot be computed in polynomial time unless $\ptime=\# \ptime$~\cite{NEURIPS2020_b1adda14}. 

\begin{proposition}\label{prop:hardness}
    Uniform-Min-$\delta$-SR$(\textsc{Linear})$ cannot be solved in polynomial time unless $\ptime = \# \ptime$.
\end{proposition}


We will consider the problem of approximating the minimum size $\delta$-SRs for linear models under the uniform distribution. To do this, let us consider two senses in which one can approximate $\delta$-SRs.
\begin{itemize}
    \item \textbf{Approximation in terms of $\delta$}: Given a linear model $\Lin$, an instance $\vx$, and a value $\delta$, we want to compute a $\delta'$-SR of size $k^\star(\Lin, \vx, \delta')$ such that $\delta'$ is close to $\delta$. Intuitively, this corresponds to the idea of stakeholders not caring about the exact value of $\delta$; e.g., $90\%$ of completions of $\vy$ agree with $\vx$ has roughly the same human implications as $89.9997\%$.
    \item \textbf{Approximation in terms of $k^\star$}: Given a linear model $\Lin$, an instance $\vx$, and a value $\delta$, we want to compute a $\delta$-SR of size $k$ such that $k$ is not much bigger than $k^\star(\Lin, \vx, \delta)$. Intuitively, even though stakeholders want \emph{small} explanations, it is not required to find the smallest possible explanation.
\end{itemize}

Furthermore, we define a slighty different sense in which the value of $\delta$ can be relaxed:

\begin{definition}[$(\delta, \varepsilon)$-relaxed minimum SR]
    Given a model $\M$, an instance $\vx$, and values $\delta, \varepsilon \in (0, 1)$, we say a partial instance $\vy$ of size $k$ is a $(\delta, \varepsilon)$-relaxed minimum SR if the two following conditions hold:
    \begin{enumerate}
        \item The partial instance $\vy$ is close enough to a $\delta$-SR:
        \[
            \Pr_{\vz \sim \D_{\vy}}\big[\M(\vz) = \M(\vx)\big] \geq \delta - \varepsilon.
        \]
        \item No smaller partial instance $\vy'$ is a clear candidate to be a $\delta$-SR:
        \[
            \Pr_{\vz \sim \D_{\vy'}}\big[\M(\vz) = \M(\vx)\big] < \delta - 2\varepsilon, \quad \forall \vy', |\vy'|_e < |\vy|_e.
        \]
    \end{enumerate}
\end{definition}


With the previous notions, we can state two results and two open problems.
\begin{theorem}\label{thm:delta-approximation}
    The Uniform-Min-$\delta$-SR$(\textsc{Linear})$ problem admits an FPRAS with respect to $\delta$. In particular, there exists an algorithm that computes a minimum $\delta'$-SR for some $\delta' \in [\delta, \delta + \varepsilon]$, and succeeds with probability at least $1-\gamma$, running in time $\tilde{O}\left(\frac{d^3}{\varepsilon^2\gamma^2}\right)$.
\end{theorem}

\begin{theorem}\label{thm:delta-relaxed}
    We can compute a $(\delta, \varepsilon)$-relaxed minimum SR in polynomial time for linear models under the uniform distribution, with a runtime of 
    \[ 
        \tilde{O}\left(\frac{d}{\varepsilon^2\gamma^2}\right).
    \]
\end{theorem}

\begin{open}
    Is there an FPRAS with respect to $k^\star$ for  Uniform-Min-$\delta$-SR$(\textsc{Linear})$?
\end{open}

\begin{open}
    Is there an FPRAS with respect to $\delta$ for Product-Min$\delta$-SR$(\textsc{Linear})$? That is, under product distributions in which each feature $i$ has an associated probability $p_i$ of taking value $1$ and these are all independent.
\end{open}

% \begin{theorem}\label{thm:k-approximation}
%     The Uniform-Min-$\delta$-SR$(\textsc{Linear})$ problem can be approximated over $k^\star$ with additive error $1$ and probability of success $1-\gamma$ in time $\tilde{O}\left( \frac{d^2}{\varepsilon^2\gamma^2}\right)$.
% \end{theorem}

In order to prove~\Cref{thm:delta-approximation,thm:delta-relaxed} we will need two main ideas: first, the fact that we can estimate the probabilities of linear models accepting a partial instance through sampling, and second, that under the uniform distribution it is easy to decide which features ought to be part of small explanations.

\subsection{Estimating the Probability of Acceptance}
 The  hardness of computing
$\Pr_{\vz  \in \D(\vy)}[\M(\vz) = 1]$ is about computing it to arbitrarily high precision, i.e., with an additive error within $O(2^{-d})$. However, computing a less precise estimation of $\Pr_{\vz \in \D(\vy)}[\M(\vz) = 1]$ is simple, as the next fact (which is a direct consequence of Hoeffding's inequality)  states.

\begin{fact}\label{fact:hoeffding}
    Let $f$ be an arbitrary boolean function on $n$ variables. Let $M$ be any positive integer,
    and let $\vx_1, \ldots, \vx_M$ be $M$ uniformly random samples from $\{0, 1\}^n$. Then 
    \[
        \hat{\mu} := \frac{\sum_{i=1}^M [f(\vx_i) = 1]}{M}
    \]
    is an unbiased estimator for 
    \[
        \mu := \Pr_{\vx \in \{0, 1\}^n}[f(\vx) = 1],
    \]
    and 
    \[
    \Pr[\left|\hat{\mu} - \mu \right| \leq t] \geq 1 - 2e^{-2t^2 M},
    \]
    which is at least $1 - \gamma$ for $M = \frac{1}{2t^2} \log(2/\gamma)$.
\end{fact}

 \paragraph{Smoothed Explanations} As a consequence of the previous idea, although a minimum $\delta$-SR might be hard to compute, this crucially depends on the value of $\delta$. In the spirit of smoothed analysis, we define the computation of a min-$(\delta, \varepsilon)$-SR as follows: first, a value $\delta^\star$ is chosen uniformly at random from $[\delta-\varepsilon, \delta+\varepsilon]$, and then a min-$\delta^\star$-SR is computed. Intuitively, the idea is that as $\delta^\star$ is chosen at random, it will be unlikely that a value that makes the computation hard is chosen. 

 We will prove the following proposition:

\begin{proposition}
    \label{prop:smoothed-explanation}
    Given a linear model $\Lin$ and an input $\vx$, we can compute a $(\delta, \varepsilon)$-SR successfully with probability $1 - \gamma$ in time polynomial in $d$, $1/\varepsilon$ and $1/\gamma$. In particular, in time $\tilde{O}\left( \frac{d^2}{\varepsilon^2\gamma^2}\right)$.
\end{proposition}

Note that~\Cref{prop:smoothed-explanation} immediately gives us~\Cref{thm:delta-approximation}, as we can set $\delta' =  \delta+\varepsilon/2$ and $\varepsilon' = \varepsilon/2$, which means with probability $1 - \gamma$ we will obtain a $(\delta', \varepsilon')$-SR, whose probability guarantee is in 
\[
  [\delta' - \varepsilon', \delta' + \varepsilon'] =  [\delta, \delta + \varepsilon].
\]

Furthermore, if one were to prefer explanations that err on the opposite side, meaning that their probability is in $[\delta - \varepsilon, \delta]$, our results would hold the same, by simply setting $\delta' = \delta - \varepsilon/2$, $\varepsilon' = \varepsilon/2$, and naturally these would be guaranteed to be smaller in size that the minimum $\delta$-SR.

Before proving~\Cref{prop:smoothed-explanation}, we need to prove a lemma concerning the easiness of selecting the features of the desired explanation.

\subsection{Feature Selection}

Even if we were granted an oracle computing the probabilities
\(
    \Pr_{\vz  \in \D(\vy)}[\M(\vz) = 1]
\), that would not be necessarily enough to efficiently compute the minimum $\delta$-SR. Indeed, for decision trees, the counting problem can be easily solved in polynomial time~\cite{NEURIPS2020_b1adda14}, and yet the computation of $\delta$-SRs of minimum size is hard, even to approximate~\cite{Arenas_Barcelo_Romero_Subercaseaux_2022,Kozachinskiy_2023}.
Intuitively, the problem for decision trees is that, even if we were told that the minimum $\delta$-SR has exactly $k$ features, it is not obvious how to search for it better than enumerating all $\binom{d}{k}$ subsets. The case of linear models, however, is different, at least under the uniform distribution. In this case, every feature $i$ that is not part of the explanation will take value $0$ or $1$ independently with probability $\nicefrac{1}{2}$, and contribute to the classification according to its weight $w_i$. In other words, we can sort the features according to their weights (with some care about signs), and select them greedily to build a small $\delta$-SR. A proof for the deterministic case ($\delta = 1$) was already given in~\cite{NEURIPS2020_b1adda14} and sketched earlier on by~\cite{ExplainingNaiveBayes}.

\begin{definition}\label{def:scores}
    Given a linear model $\Lin = (\vw, t)$, and an instance $\vx$, both having dimension $d$, we define the \emph{score} of feature $i \in [d]$ as 
    \[
        s_i := w_i \cdot (2x_i - 1) \cdot (2\left(\Lin(\vx) - 1\right)).    
    \]
\end{definition}

In other words, the sign of $s_i$ is $+1$ if the feature is ``helping'' the classification, and $-1$ if it is ``hurting'' it. The magnitude of $s_i$ is proportional to the weight of the feature $i$. Changing the value of feature $i$ in an instance $\vx$ would decrease $\vw \cdot \vx$ by $s_i$ if $\Lin(\vx) = 1$, and increase it by $s_i$ if $\Lin(\vx) = 0$.
For the uniform distribution, we can prove the following lemma that basically states that, if we are promised a $\delta$-SR of size $k$ exists, then one can be found by taking the top-$k$ features of maximum score $s_i$.

\begin{lemma}\label{lemma:greedy}
Given a linear model $\Lin$, an instance $\vx$,  a value $\delta \in (0, 1]$, and an integer $k$, then there exists a $\delta$-SR for $\vx$ under $\Lin$ of size $k$ if and only if the partial instance defined only on the $k$ features of $\vx$ with maximum score is a $\delta$-SR for $\vx$ under $\Lin$.
\end{lemma}

Even though a proof of~\Cref{lemma:greedy} is presented in the appendix, let us provide a self-contained example that should convince a reader of the veracity of the lemma.
\begin{example}\label{ex:greedy}
 Consider an instance $\vx = (1,\, 0, \, 0, \, 1, \, 1)$ and the linear model $\Lin$ be defined by 
 \[ 
    \vw = (5, \, 1, \, -3,\, 2, -1) \quad ; \quad t = 5.
 \]
 It is easy to check that $\vw \cdot \vx = 5$, and thus $\Lin(\vx) = 1$. The feature scores, according to~\Cref{def:scores}, are:
 \[
  s_1 = 5, \; s_2 = -1, \; s_3 = 3, \; s_4 = 2, \; s_5 = -1.
 \]
Consider the partial instances $\vy^{(1)} = (\bot, \, 0, \, 0, \, 1, \, 1)$ and $\vy^{(2)} = (1, \, \bot, \, 0, \, 1, \, 1)$. The instance $\vx$ is a completion of both $\vy^{(1)}$ and $\vy^{(2)}$, but $\vy^{(1)}$ also has completion 
\[
    \vx^{(1)} = (0, \, 0, \, 0, \, 1, \, 1),
\]
whereas $\vy^{(2)}$ has also completion 
\[
    \vx^{(2)} = (1, \, 1, \, 0, \, 1, \, 1).
\]
Note that $\vw \cdot  \vx^{(1)} = 1 = \vw \cdot \vx - s_1$, whereas $\vw \cdot  \vx^{(2)} = 6 = \vw \cdot \vx - s_2$. Intuitively, this means that it is better to keep feature $1$ as part of the explanation, but not feature $2$. If we want an explanation with only two features, we should choose feature $1$ and feature $3$, as they have the highest scores. Indeed,~\Cref{table:ex-greedy} presents the probabilities to all possible explanations of size $2$.
\begin{table}
    \caption{Table of probabilities associated to~\Cref{ex:greedy}.}\label{table:ex-greedy}
    \centering
    \begin{tabular}{rrr}
        \toprule
        Partial instance & Features included & Probability\\ \midrule 
    $(1, \,\bot, \,0, \,\bot, \,\bot)$ & $\{1, \,3\}$ & $\nicefrac{ 7 }{ 8 }$\\
    $(1, \,\bot, \,\bot, \,1, \,\bot)$ & $\{1, \,4\}$ & $\nicefrac{ 5 }{ 8 }$\\
    $(\bot, \,\bot, \,0, \,1, \,\bot)$ & $\{3, \,4\}$ & $\nicefrac{ 1 }{ 2 }$\\
    $(\bot, \,\bot, \,0, \,\bot, \,1)$ & $\{3, \,5\}$ & $\nicefrac{ 3 }{ 8 }$\\
    $(\bot, \,0, \,0, \,\bot, \,\bot)$ & $\{2, \,3\}$ & $\nicefrac{ 3 }{ 8 }$\\
    $(1, \,\bot, \,\bot, \,\bot, \,1)$ & $\{1, \,5\}$ & $\nicefrac{ 3 }{ 8 }$\\
    $(1, \,0, \,\bot, \,\bot, \,\bot)$ & $\{1, \,2\}$ & $\nicefrac{ 3 }{ 8 }$\\
    $(\bot, \,\bot, \,\bot, \,1, \,1)$ & $\{4, \,5\}$ & $\nicefrac{ 1 }{ 4 }$\\
    $(\bot, \,0, \,\bot, \,1, \,\bot)$ & $\{2, \,4\}$ & $\nicefrac{ 1 }{ 4 }$\\
    $(\bot, \,0, \,\bot, \,\bot, \,1)$ & $\{2, \,5\}$ & $\nicefrac{ 1 }{ 8 }$\\
        \bottomrule
    \end{tabular}
\end{table}
\end{example}

% \todo[inline]{Bernardo: The proof of this is trivial but annoying, it suffices to do an exchange argument. I will write it down later.}

With~\Cref{lemma:greedy} in hand, we can now prove~\Cref{prop:smoothed-explanation}.

\renewcommand{\algorithmiccomment}[1]{\; \; \texttt{/* #1 */}}
\begin{algorithm}[tb]
	\caption{Uni-Linear MonteCarlo}
	\label{alg:algorithm}
	\textbf{Input}: Linear model $\Lin$, instance $\vx$, parameters $\delta \in (0, 1)$\\
	\textbf{Parameter}:  $\varepsilon \in (0, 1)$,  $\gamma \in (0, 1)$\\
	\textbf{Output}: A value $\delta^\star \in [\delta-\varepsilon, \delta+\varepsilon]$ together with an optimal $\delta^\star$-SR explanation for $\vx$.\\
	\begin{algorithmic}[1] %[1] enables line numbers
	\STATE Sample $\delta^\star$ uniformly at random from $[\delta-\varepsilon, \delta + \varepsilon]$
	\STATE Compute the score $s_i$ for each feature
	\STATE Let $\vec{\mathcal{F}}$ be the sequence of pairs $(i, s_i)$ sorted decreasingly by $s_i$
	\STATE Let $\ell$ be the max value such that $\vec{\mathcal{F}}[\ell] = (\lambda, s_\lambda)$ has $s_\lambda > 0$.
	\STATE For $k \in \{1, \ldots, \ell\}$, let $\vy^{(k)} \subseteq \vx$ be the partial instance defined only in features $\vec{\mathcal{F}}[1][1], \ldots, \vec{\mathcal{F}}[k][1]$.
	\STATE Let $\textrm{LB} = 0$, and $\textrm{UB} = \ell$.
	\STATE $M \gets \frac{2\log d^2}{\varepsilon^2 \gamma^2} \log(2 \log d / \gamma)$.

	\WHILE[Binary Search]{$\textrm{LB} \neq \textrm{UB}$} 
	% \COMMENT{Binary Search}
		\STATE $m \gets \left(\textrm{LB} + \textrm{UB} \right)/2$.
		\STATE $\hat{v}_m \gets \textsc{MonteCarloSampling}(\Lin, \vy^{(m)}, M)$.
		\IF {$\hat{v}_m \geq \delta^\star$}
			\STATE $\textrm{UB} \gets m$
		\ELSE
			\STATE $\textrm{LB} \gets (m+1)$
		\ENDIF
	\ENDWHILE
	\STATE $k^\star \gets \textrm{LB}$ (or equivalently, $\textrm{UB}$)
	\STATE \RETURN $(\delta^\star, \vy^{(k^\star)})$
	\end{algorithmic}
	\end{algorithm}


\begin{algorithm}[tb]
	\caption{MonteCarloSampling}	\label{alg:montecarlo}
	\textbf{Input}: Linear model $\Lin$, instance $\vx$, number of samples $M$\\
	\textbf{Output}: An estimate~$\hat{v}$~of~$\Pr_{\vz \in \mathcal{U}_\vx}[\Lin(\vz)=1]$.\\
	\begin{algorithmic}[1]
	\STATE $\hat{v} \gets 0$
	\FOR{$i = 1$ to $M$}
		\STATE Sample $\vz \sim \textsc{Comp}(\vx)$
		\STATE $\hat{v} \gets \hat{v} + \Lin(\vz)$
	\ENDFOR
	\STATE $\hat{v} \gets \hat{v}/M$
	\STATE \RETURN $\hat{v}$
\end{algorithmic}
\end{algorithm}


	

\begin{proof}[Proof of~\Cref{prop:smoothed-explanation}]
We use~\Cref{alg:algorithm}. Let us define the values $v_k$ as
	\[
		v_k := \Pr_{\vz \in \textsc{Comp}(\vy^{(k)})}[\Lin(\vz) = 1].
	\]
As a result of the binary search (lines 8-16), the algorithm obtains $k^\star$, the smallest $k$ such that
\[
	\hat{v}_k \geq \delta^\star,
\]
and our goal is to show that with good probability $k^\star$ is also the smallest $k$ such that $v_k \geq \delta^\star$, which would imply the correctness of the algorithm.
Let $S$ be a random variable corresponding to the set of values $k$ such that~\Cref{alg:algorithm} enters line $10$ with $m=k$, and note that if for every $k$ in $S$ it happens that the events 
\[
	A_k := \left(v_k \geq \delta^\star \right) \text{ and }  B_k := \left(\hat{v_k}(M) \geq \delta^\star\right)
\]
are equivalent (i.e., either both occur or neither occurs), then the algorithm will succeed, as that would indeed imply that $k^\star$ is the smallest $k$ such that $v_k \geq \delta^\star$. 

Then, define events $E_k$ and $F_k$ as follows:
\begin{align*}
	E_k &:= |\delta^\star - v_k| > \frac{\epsilon \gamma}{\log d},\\
	F_k &:= k \in S \land |\hat{v_k}(M) - v_k| \leq \frac{\epsilon \gamma}{\log d}.
\end{align*}
We claim that if both $E_k$ and $F_k$ hold for some $k$, then $A_k$ and $B_k$ are equivalent events. Indeed,
\begin{align*}
	A_k &\iff v_k \geq \delta^\star\\
		&\iff v_k \geq \delta^\star + \frac{\epsilon \gamma}{\log d} \tag{By $E_k$}\\
		&\iff v_k - \frac{\epsilon \gamma}{\log d} \geq \delta^\star\\
		&\iff \hat{v_k}(M) \geq \delta^\star \tag{By $F_k$}\\
		&\iff B_k.
\end{align*}

Thus, if we show that $E_k$ and $F_k$ hold with good probability for every $k \in S$, we can conclude the theorem.
To see that, note first that for every $k \in [d]$, line 1 implies
\[
	\Pr[\overline{E_k}] = \Pr\left[\delta^\star \in \left[v_k - \frac{\epsilon \gamma}{\log d}, v_k + \frac{\epsilon \gamma}{\log d}\right]\right] \leq \frac{\frac{2\epsilon \gamma}{\log d}}{2\epsilon} = \frac{\gamma}{\log d}.
\]

It is tempting to say that we have 
\[ 
	\Pr\left[\bigcap_{k \in S} E_k\right] = 1 -  \Pr\left[\bigcup_{k \in S} \overline{E_k} \right] \geq 1 - \gamma,
\]
through a union bound on the $\log d$ elements of $S$\footnote{For simplicity we will say $|S| \leq \log d$, even though the exact bound for a binary search is $|S| \leq \lfloor \log d  + 1 \rfloor$.}. However, $S$ itself is a random variable, and $E_k$ and $k \in S$ are not (necessarily) independent events, so we need to be more careful. 
Using the law of total probabilities, we have 
\begin{align*}
	\Pr\left[\bigcap_{k \in S} E_k\right] &= \sum_{S' \subseteq [d]} \Pr\left[S = S'  \mid \bigcap_{k \in S'} E_k \right] \Pr\left[\bigcap_{k \in S'} E_k\right]\\
	&= \sum_{\substack{S' \subseteq [d]\\ |S'| \leq \log d}} \Pr\left[S = S' \mid \bigcap_{k \in S'} E_k\right] \Pr\left[\bigcap_{k \in S'} E_k\right],
\end{align*}
where we can now effectively use the union bound to say that for any fixed $S'$ with $|S'| \leq \log d$ we have
\[ 
	\Pr\left[\bigcap_{k \in S'} E_k \right] \geq 1 - \gamma,
\]
from where we get
\begin{align*}
	\Pr\left[\bigcap_{k \in S} E_k\right] &=  \sum_{\substack{S' \subseteq [d]\\ |S'| \leq \log d}} \Pr\left[S = S'\mid \bigcap_{k \in S'} E_k\right] \Pr\left[\bigcap_{k \in S'} E_k\right]\\
	&\geq (1-\gamma) \sum_{\substack{S' \subseteq [d]\\ |S'| \leq \log d}} \Pr\left[S = S' \mid \bigcap_{k \in S'} E_k\right]\\
	&= 1 - \gamma.
\end{align*}
 Let us now argue that $F_k$ holds with good probability for every $k \in S$. Indeed, we have for any $k$ that
 \begin{align*}
 \Pr\left[ \, \overline{F_k} \mid k \in S \right] &= \Pr\left[|\hat{v_k}(M) - v_k| > \frac{\epsilon \gamma}{\log d}\right]\\
 	&\leq \frac{\gamma}{\log d}, \tag{By~\Cref{fact:hoeffding} and line 7.}
 \end{align*}
from where a final union bound (using the same trick as for $E_k$) yields 
\begin{align*}
	\Pr\left[\bigcap_{k \in S} F_k\right] &= 1 - \Pr\left[\bigcup_{k \in S} \overline{F_k}\right]\\
	&\geq 1 - \sum_{k \in S} \Pr\left[\overline{F_k} \mid k \in S\right]\\
	&\geq 1 - \sum_{k \in S} \frac{\gamma}{\log d}\\
	&= 1 - \gamma.
\end{align*}
Therefore, the algorithm will succeed with probability at least 
\[ 
	\Pr\left[\bigcap_{k \in S} E_k\right] \cdot \Pr\left[\bigcap_{k \in S} F_k\right] \geq (1-\gamma)^2 \geq 1-2\gamma.
\]
The runtime is simply
$O(\log d \cdot M \cdot d  )$; as (i) the binary search performs $O(\log d)$ steps; (ii) each of the binary search steps requires $M$ samples, and (iii) each sample requires evaluating the model $\Lin$ and thus takes time $O(d)$. Naturally, running the algorithm with $\gamma' = 1/2 \cdot \gamma$ will yield a success probability of $1-\gamma$ without changing the asymptotic runtime, and thus we conclude the proof.


% We thus want to lower bound the probability of some decently large value $\Delta$ such that $|\delta^\star - v_k| \geq \Delta$ for every $k \in [d]$; that way, it will be enough to estimate the values $v_k$ up to an additive error of $\Delta/2$. Given that there are $d$ values $v_
% k$, and each of them forbids a segment of size $2\Delta$ for $\delta^\star$, then the probability of choosing a value of $\delta^\star$ such that $|\delta^\star - v_k| \geq \Delta$ for every $k \in [d]$ is at least
% \[
% 	1 - \frac{d\Delta}{\epsilon}.
% \]
% As we want our algorithm to succeed with probability $1 - \gamma$, that requires 
% \(
% \frac{d\Delta}{\epsilon} \leq \gamma
% \)
% and thus
% \[
% 	\Delta \leq \frac{\epsilon \gamma}{d}.
% \]
% Thus we need to sample with an additive error smaller than
% \(
% \frac{\epsilon}{2d \gamma},
% \)
% which requires the number of samples $M$ to be such that
% \[
% 	\frac{\log M}{\sqrt{M}} \leq \frac{\epsilon\gamma}{2d}.
% \] 
% %%Considering that
% %%\[
% %%	\frac{\log M}{\sqrt{M}} \leq \frac{\log M}{\log M \sqrt[2+\gamma]{M}} = \frac{1}{\sqrt[2+\gamma]{M}},
% %%\]
% %Let us try to simplify things by considering establishing
% %\[
% %	\frac{1}{\sqrt{M'}} \leq \frac{\epsilon\gamma}{2d}, 
% %\]
% %or equivalently,
% %\[
% %	M \geq \left( \frac{2d}{\epsilon\gamma} \right)^2.
% %\]
% Now we pose
% \[
% 	M =	9\cdot \left( \frac{2d}{\epsilon\gamma} \right)^2\log\left(\frac{2d}{\epsilon\gamma}\right)^2,
% \]
% so that $\log M \leq 3\log\left(\frac{2d}{\epsilon\gamma}\right)$, and thus
% \[
% 	\frac{\log M}{\sqrt{M}} \leq \frac{3\log(2/(\epsilon\gamma))}{3\left( \frac{2d}{\epsilon\gamma} \right)\log\left(\frac{2d}{\epsilon\gamma}\right)} = \frac{\epsilon\gamma}{2d},
% \]
% as desired.

% Thus, the algorithm above, which requires a sampling process at most $\log d$ times, using binary search to find the min, can be carried out with probability of success at least $1-\gamma$. in time 
% \[
% O(\log d \cdot M) = \tilde{O}\left(\left(\frac{d}{\epsilon\gamma}\right)^2\right).
% \]


% \begin{figure}
% 	\centering
% 	\scalebox{0.8}{
% 	\begin{tikzpicture}
% 		\node[] (beg) at (0, 0) {};
% 		\node[] (end) at (11, 0) {};
% 				\node[] (a) at (10, 0.5) {\small $1$};
% 		\node[] (amark) at (10, 0) {\small $\mid$};
% 		\node[] (b) at (5, 0.5) {\small $\nicefrac{1}{2}$};
		
% 		\node[] (b) at (0, 0.5) {\small $0$};
		
% 		\node[] (bmark) at (5, 0) {\small $\mid$};
		
% 		\node[] (leftPar) at (5.5, 0) {\textcolor{blue}{$\Big[$}};
% 		\node[] (rightPar) at (9.5, 0) {\textcolor{blue}{$\Big]$}};
		
% 		\draw[-, blue, dashed, very thick]  (leftPar) -- (rightPar);
		
% 		\draw[|-, very thick] (beg) -- (leftPar);
% 			\draw[->, very thick] (rightPar) -- (end);

		
% 		\node[] (left) at (5.5, -0.7) {\textcolor{blue}{\scriptsize $\delta-\varepsilon$}};
% 		\node[] (right) at (9.5, -0.7) {\textcolor{blue}{\scriptsize $\delta+\varepsilon$}};
		
% 		\node[] (delta) at (7.5, -0.5) {\textcolor{blue}{$\delta$}};
		
% 		\node[] (dmark) at (7.5, 0) {\textcolor{blue}{\small $\mid$}};
		
		
% 		\node[] (v1) at (8.8, 0) {\textcolor{purple}{\small $\mid$}};
% 		\node[] (v1) at (8.8, -0.4) {\textcolor{purple}{\scriptsize $v_1$}};
		
% 		\node[] (v2) at (6.6, 0) {\textcolor{purple}{\small $\mid$}};
% 		\node[] (v2) at (6.6, -0.4) {\textcolor{purple}{\scriptsize $v_2$}};
		
% 		\node[] (v4) at (3.6, 0) {\textcolor{purple}{\small $\mid$}};
% 		\node[] (v4) at (3.6, -0.4) {\textcolor{purple}{\scriptsize $v_3$}};
		
% 		\node[] (v3) at (1.2, 0) {\textcolor{purple}{\small $\mid$}};
% 		\node[] (v3) at (1.2, -0.4) {\textcolor{purple}{\scriptsize $v_4$}};
		
% 		\draw[rectangle, fill=purple, line width=0pt] (6.3, -0.2) -- (6.3, 0.2) -- (6.9, 0.2) -- (6.9, -0.2) -- (6.3, -0.2);
		
% 		\draw[rectangle, fill=purple, line width=0pt] (8.5, -0.2) -- (8.5, 0.2) -- (9.1, 0.2) -- (9.1, -0.2) -- (8.5, -0.2);
		
		
% 		\node at (6.6, 0.65) {$\Delta$};
% 		\node at (8.8, 0.65) {$\Delta$};
% 		\draw[decorate, decoration={brace, amplitude=3pt}, thick] (6.3, 0.28) -- (6.9, 0.28);
		
% 			\draw[decorate, decoration={brace, amplitude=3pt}, thick] (8.5, 0.28) -- (9.1, 0.28);
		
% 		\node[] (deltastar) at (8.1, -1.2) {\large \textcolor{violet}{$\delta^\star$}};
% 		\draw[-, , violet, thick] (8.0, -0.9) -- (8.0, 0.9);

 
% 	\end{tikzpicture}
% 	}
% 	\caption{Illustration of the proof, using $d=4, \delta = \nicefrac{3}{4}$ as an example. In this case, the probability of failure is $\frac{2\Delta}{\varepsilon}.$}
% \end{figure}

\end{proof}


% Let us restate and prove our other theorem.

% \begin{theorem}
%     The Uniform-Min-$\delta$-SR$(\textsc{Linear})$ problem can be approximated over $k^\star$ with additive error $1$ and probability of success $1-\gamma$ in time $\tilde{O}\left( \frac{d^2}{\varepsilon^2\gamma^2}\right)$.
% \end{theorem}
% \begin{proof}[Proof sketch]
%     Sort the features according to \emph{score} again, then we try to find the minimum $k$ such that the top-$k$ most important features give a probability of $\delta$. In particular, we will accept the first $k$ such that the empirical probability is above $\delta + (1-\delta)/2d$. The chances of the ``true'' probability being above $\delta$ will naturally be good, the problem is to guarantee that we won't overshoot by much. The main insight here is that, if $k^*$ is the true optimal answer (thus having probability at least $\delta$), then $k^*+1$ should always have the desired probability; note that the remaining probability is $1-\delta$, and there are $d-{k^*}$ features remaining, so the next most important one must yield at least a gain of 
%     $\frac{1-\delta}{d - k^*} \geq \frac{1 - \delta}{d}$. So by sampling with an error of $\frac{1-\delta}{2d}$ we will recognize this, which by using the fact requires 
% $
%     \tilde{O}\left(\left(\frac{1-\delta}{2d}\right)^2\right)
% $ many samples to ensure with high probability. Note that the problematic case is now when $\delta \approx 1$. 
% \end{proof}


Due to the complexity of finding even subset-minimal $\delta$-SR,~\citet{izza2024locallyminimalprobabilisticexplanations} have proposed to study ``locally minimal'' $\delta$-SR, which are $\delta$-SRs such that the removal of any feature from the explanation would decrease its probabilistic guarantee below $\delta$.
Interestingly, we can generalize a proof from~\cite{NEURIPS2022_b8963f6a} to show that, over lineal models even in the more general case of product distributions (distributions over $\{0,1\}^d$ that are products of independent Bernoulli variables of potentially different parameters), every locally minimal $\delta$-SR is a subset-minimal $\delta$-SR. This allows leveraging the previous results of~\citet{izza2024locallyminimalprobabilisticexplanations} to subset-minimal $\delta$-SRs in the case of linear models.


\begin{theorem}\label{thm:locally-minimal}
For linear models, under any product distribution, every locally minimal $\delta$-SR is a subset-minimal $\delta$-SR.
\end{theorem}
\begin{proof}[Proof sketch]
    Define the \emph{``locality''} gap $\textsc{lgap}(\vy)$ of a locally minimal $\delta$-SR $\vy$ as the smallest value $g$ such that $|\vy^\star|_\bot - |\vy|_\bot = g$ for some $\vy^\star \subseteq \vy$ that is a $\delta$-SR.
    If $g = 0$, then $\vy$ is globally minimal, and we are done. If $g$ were to be $1$, then $\vy$ would not be locally minimal, a contradiction. Therefore, we can safely assume $g \geq 2$ from now on.
    Let $\Lin, \vy$ be such that $\vy$ is locally minimal $\delta$-SR and $\textsc{lgap}(\vy) \geq 2$. We will find a contradiction by the following method:
    \begin{itemize}
        \item Let $\vy^\star$ be the $\delta$-SR such that $| \vy \setminus \vy^\star | = \textsc{lgap}(\vy)$.
        \item Every feature in $\vy \setminus \vy^\star$ is either ``good'', if its score is positive, or ``bad'' if its score is negative.
        \item Fix any feature $i$ in $\vy \setminus \vy^\star$. If $i$ is good, then $\vy^\star \oplus i$, meaning the partial instance obtained by taking $\vy$ and setting its $i$-th feature to $x_i$, has a probability guarantee greater or equal than that of $\vy^\star$ (the proof of this fact is very similar to the proof of~\Cref{lemma:greedy}), and the gap has reduced.
                On the other hand, if $i$ is bad, then $\vy \ominus i$, meaning the partial instance obtained from $\vy$ by setting $y_i = \bot$, has greater-equal probability than $\vy$, contradicting the fact that $\vy$ is locally minimal.
    \end{itemize}


\end{proof}


\bibliography{references}

\newpage
\onecolumn
\appendix
\section*{Appendix}


\section{Calculation for~\Cref{ex:delta-sr-size}}
Consider the following version of Chernoff bound.
\begin{lemma}[Chernoff bound]
    Let $X$ be a finite sum of independent Bernoulli variables, with $\E[X] = \mu > 0$. Then for any $t \geq 0$ we have
    \[
    \Pr \Big[ \left|X - \mu\right| \geq t \Big] \leq 2\exp\left(\frac{-t^2}{3 \mu} \right).
    \] 
    
    \label{lemma:chernoff}  
    \end{lemma}
If we define the Bernoulli variables $X_i := (z_i = 1)$ for $\vz \sim \U(\vy)$ (the uniform distribution over $\comp(\vy)$), then the variables $X_2, \ldots, X_{1000}$ are identical independent Bernoulli variables, and if $X = \sum_{i = 2}^{1000} X_i$ we have 
\[
    \mu := \mathbb{E}[X] =  999\mathbb{E}[X_2] = \frac{999}{2},
\]
as each $X_i$ has expectation $\frac{1}{2}$ becuase $\U$ is the uniform distribution. Then, using \Cref{lemma:chernoff}, we have that 
\begin{align*}
    \Pr\left[ X < 250 \right] &\leq 2 \exp\left(\frac{-(445-250)^2    \cdot 2}{3 \cdot 999}\right)\\ &\approx 2 \exp(-32.29...)\\ &< 2 \cdot 10^{14}.
\end{align*}
To conclude, note that
\begin{align*}
    \Pr_{\vz \sim \U(\vy)}
     \left[\Lin(\vz) = 1 \right] 
      &= \Pr_{\vz \sim \U(\vy)}
    \left[\vz \cdot \vw \geq 1250 \right]\\
    &= \Pr_{\vz \sim \U(\vy)}\left[1000 + \sum_{i=2}^{1000}{z_i} \geq 1250\right]\\
    &= 1 - \Pr\left[\sum_{i=2}^{1000}{z_i} < 250\right]\\
    &= 1 - \Pr\left[ X < 250 \right] > 1 - 2\cdot 10^{14} > 0.999999.
\end{align*}
    

\section{Missing Proofs}

\subsection{\Cref{prop:hardness}}
\begin{proof}[Proof of~\Cref{prop:hardness}]
    The proof is a twist on ~\citet[Lemma 28]{NEURIPS2020_b1adda14}; let $(s_1, \ldots, s_n, T) \in \mathbb{N}^{n+1}$ be an instance of the $\# \ptime$-complete problem $\#\Knapsack$, that consists on counting the number of sets $S \subseteq \{s_1, \ldots, s_n\}$ such that $\sum_{s \in S}s \leq T$.  
    We can assume that $\sum_{i=1}^n s_i > T$, as otherwise the $\# \Knapsack$ instance is trivial.
    Then, let $\Lin$ be a linear model with weights $w_i = s_i$, and threshold $t = T+1$.
    Now, consider the problem of deciding whether $\#\Knapsack(s_1, \ldots, s_n, T) \geq m$ for an input $m$, which cannot be solved in polynomial time unless $\ptime = \# \ptime$.
    Let $\vx = (1, 1, \ldots, 1)$, and $\delta = \frac{m}{2^{n}}$. We claim that $(\Lin, \vx, \delta, k=0)$ is a Yes-instance for Uniform-Min-$\delta$-SR$(\textsc{Linear})$ if and only if $\#\Knapsack(s_1, \ldots, s_n, T) \geq m$. 
    First, note that $\Lin(\vx) = 1$, since $\sum_{i=1}^n w_i x_i = \sum_{i=1}^n s_i \geq T+1 = t$. 
    For every set $S \subseteq \{s_1, \ldots, s_n\}$ such that $\sum_{s \in S}s \leq T$, its complement $\overline{S} := \{s_1, \ldots, s_n\} \setminus S$ holds $\sum_{s \in \overline{S}}s > T$, and as all values are integers, this implies as well
    \[
        \sum_{s \in \overline{S}} s \geq T+1 = t.
    \]
    To each such set $\overline{S}$, we associate the instance $\vz(\overline{S})$ defined as
    \[
        z(\overline{S})_i = \begin{cases}
            1 & \text{if } s_i \in \overline{S}\\
            0 & \text{otherwise}.
        \end{cases}
    \]
    Now note that 
    \[
        \Lin(\vz(\overline{S})) = \begin{cases}
            1 & \text{if } \sum_{s_i \in \overline{S}} w_i \geq T+1\\
            0 & \text{otherwise} \end{cases} = 1,
    \]
    and thus there is a bijection between the sets $S$ whose sum is at most $T$ and the instances $\vz(\overline{S})$ such that $\Lin(\vz(\overline{S})) = 1 = \Lin(\vx)$.
    To conclude, simply note that the previous bijection implies
    \[
        \Pr_{\vz \sim \U(\bot^d)}\Big[\Lin(\vz) = 1\Big] = \frac{\#\Knapsack(s_1, \ldots, s_n, T)}{2^n},
    \] 
    and thus the ``empty explanation'' $\bot^d := (\bot, \bot, \ldots, \bot)$ has probability at least $\delta$ if and only if $\#\Knapsack(s_1, \ldots, s_n, T) \geq m$. As the empty explanation is the only one with size $\leq 0$, we conclude the proof.
    

%     First, recall that the problem of counting the number of ``positive completions'' of a partial instance $\vy$ is $\# \ptime$-hard for linear models~\cite{NEURIPS2020_b1adda14}; that is, given a partial instance $\vy$, a linear model $\Lin$ and a positive integer $K$, it is $\mathrm{PP}$-hard to determine whether there are at least $K$ instances $\vz \in \comp(\vy)$ such that $\Lin(\vz) = 1$. That result follows from the hardness of $\sharpK$.
%    Moreover, in the proof of~\cite{NEURIPS2020_b1adda14} all weights are positive $\delta = \frac{V}{2^{n - |\vy|_\bot}}.$ 
\end{proof}


\subsection{\Cref{lemma:greedy} and \Cref{thm:locally-minimal}}
    Before we prove~\Cref{lemma:greedy}, let us prove an auxiliary lemma that will also be useful for proving~\Cref{thm:locally-minimal}.
In order to state it, however, we need to define what we mean by an arbitrary product distribution. Consider Bernoulli variables $X_1, \ldots, X_d$ with probabilities $p_1, \ldots, p_d$ respectively, and let us denote by $\D = X_1 \times \cdots \times X_d$ their joint distribution, which we will call a product distribution. Then, $\D$ induces a over the set $\{0, 1\}^d$ by the rule
\[ 
    \Pr_{\vz \sim \D}[\vz = \vx] = \prod_{i=1}^d p_i^{x_i} (1-p_i)^{1-x_i}, \quad \forall x \in \{0, 1\}^d.
\]
Naturally, this induces a distribution over $\comp(\vy)$ for any partial instance $\vy$ as follows:
\[ 
    \Pr_{\vz \sim \D(\vy)}[\vz = \vx] =
        \frac{\Pr_{\vz \sim \D}[\vz = \vx]}{\sum_{\vw \in \comp(\vy)} \Pr_{\vz \sim \D}[\vz = \vx]} \quad \forall \vx \in \comp(\vy),
\]
and naturally $\Pr_{\vz \sim \D(\vy)}[\vz = \vx] = 0$ if $\vx \not\in \comp(\vy)$.

Let us introduce a last piece of notation before stating the auxiliary lemma. Given an instance $\vx$ to explain, and a partial instance $\vy \subseteq \vx$, such that $y_i = \bot$, we define the partial instance $\vy \oplus i$ by:
\[ 
    (\vy \oplus i)_j = \begin{cases}
        y_j & \text{if } j \neq i\\
        x_i & \text{otherwise}.
    \end{cases}
\]
\begin{lemma}\label{lemma:positive-score}
Let $\Lin$ be a linear model, $\vx$ an instance and $\vy \subseteq \vx$ a partial instance. Assume a product distribution $\D$. Then, if $i$ is a feature such that $y_i = \bot$, and the feature score $s_i$ holds $s_i \geq 0$, we have 
\[ 
\Pr_{\vz \in \D(\vy \oplus i)}[\Lin(\vz) = \Lin(\vx)] \geq \Pr_{\vz \in \D(\vy)}[\Lin(\vz) = \Lin(\vx)].
\]
\end{lemma}
\begin{proof}[Proof of~\Cref{lemma:positive-score}]
Let $(w_1, \ldots, w_d)$ be the weights of $\Lin$ and $t$ its threshold. Then, let us assume without loss of generality that $\Lin(\vx) = 1$, as the case $\Lin(\vx) = 0$ is analogous. 
We thus have that 
\[ 
    s_i = w_i \cdot (2x_i - 1) \geq 0,
\]
from where $s_i = w_i$ if $x_i = 1$ and $s_i = -w_i$ if $x_i = 0$.
Let us denote by $S$ the set of features $j$ such that $y_j = x_j \neq \bot$, and define 
\[ 
    t' := t - \sum_{j \in S} y_j w_j.
\]
We can then rewrite the probability of interest as
\[ 
    \Pr_{\vz \sim \D(\vy)}[\Lin(\vz) = \Lin(\vx)] =   \Pr_{\vz \sim \D(\vy)}[\Lin(\vz) = 1] = \Pr_{\vz \sim \D(\vy)}\left[\sum_{j \not\in S} z_j w_j \geq t'\right].
\]
Let us define the following two amounts:
\[ 
    \mathcal{A} := \Pr_{\vz \sim \D(\vy)}\left[\sum_{j \not\in S, j \neq i} z_j w_j \geq t' - w_i\right],
\]
\[ 
    \mathcal{B} := \Pr_{\vz \sim \D(\vy)}\left[\sum_{j \not\in S, j \neq i} z_j w_j \geq t'\right].
\]
If $p_i$ is the probability of feature $i$ under $\D$, then we have 
\[ 
    \Pr_{\vz \sim \D(\vy)}[\Lin(\vz) = \Lin(\vx)]  =  \Pr_{\vz \sim \D(\vy)}\left[\sum_{j \not\in S} z_j w_j \geq t'\right] = p_i \cdot \mathcal{A} + (1-p_i)\cdot \mathcal{B}.
\]
Now we proceed by cases on $x_i$. If $x_i = 1$, then $s_i = w_i$, and thus we know $w_i \geq 0$, from where $t' - w_i \leq t'$ and thus $\mathcal{A} \geq \mathcal{B}$. Moreover, as $x_i = 1$, we conclude 
\begin{align*}
    \Pr_{\vz \sim \D(\vy \oplus i)}[\Lin(\vz) = \Lin(\vx)] &= \mathcal{A}\\
     &= p_i \cdot \mathcal{A} + (1-p_i)\cdot \mathcal{A}\\
    &\geq p_i \cdot \mathcal{A} + (1-p_i)\cdot \mathcal{B}\\
    &= \Pr_{\vz \sim \D(\vy)}[\Lin(\vz) = \Lin(\vx)].
\end{align*}

Similarly, if $x_i = 0$, then $s_i = -w_i$, and thus we know $w_i \leq 0$, from where $t' - w_i \geq t'$ and thus $\mathcal{A} \leq \mathcal{B}$. As $x_i = 0$, we conclude 
\begin{align*}
    \Pr_{\vz \sim \D(\vy \oplus i)}[\Lin(\vz) = \Lin(\vx)] &= \mathcal{B}\\
     &= p_i \cdot \mathcal{B} + (1-p_i)\cdot \mathcal{B}\\
    &\geq p_i \cdot \mathcal{A} + (1-p_i)\cdot \mathcal{B}\\
    &= \Pr_{\vz \sim \D(\vy)}[\Lin(\vz) = \Lin(\vx)].
\end{align*}
This concludes the proof.
\end{proof}

Similarly, for any partial instance $\vy$ such that $y_i \neq \bot$, we can define the partial instance $\vy \ominus i$ as 
\[ 
    (\vy \ominus i)_j = \begin{cases}
        y_j & \text{if } j \neq i\\
        \bot & \text{otherwise}.
    \end{cases}
\]
The proof of~\Cref{lemma:positive-score}, but reversing signs, yields the following lemma. 
\begin{lemma}\label{lemma:negative-score}
    Let $\Lin$ be a linear model, $\vx$ an instance and $\vy \subseteq \vx$ a partial instance. Assume a product distribution $\D$. Then, if $i$ is a feature such that $y_i \neq \bot$, and the feature score $s_i$ holds $s_i \leq 0$, we have 
\[ 
\Pr_{\vz \in \D(\vy \ominus i)}[\Lin(\vz) = \Lin(\vx)] \geq \Pr_{\vz \in \D(\vy)}[\Lin(\vz) = \Lin(\vx)].
\]

\end{lemma}

Now, in order to prove~\Cref{lemma:greedy}, which makes two claims, we will split it into two separate lemmas. 
\begin{claim}[Part 1 of~\Cref{lemma:greedy}]\label{lemma:greedy-1}
    Given a linear model $\Lin$, and an instance $\vx$, if $\vy^{(1)}, \ldots, \vy^{(d)}$ are the partial instances of $\vx$ such that $\vy^{(k)} \subseteq \vx$ is defined only in the top $k$ features of maximum score, then
    \[ 
        \Pr_{\vz \sim \U(\vy^{(k+1)})}[\Lin(\vz) = \Lin(\vx)] \geq \Pr_{\vz \sim \U(\vy^{(k)})}[\Lin(\vz) = \Lin(\vx)]
    \]
    for all $k \in \{1, \ldots, d-1\}$, and naturally, 
    \[ 
    \Pr_{\vz \sim \U(\vy^{(d)})}[\Lin(\vz) = \Lin(\vx)] = 1.
    \]
\end{claim}
\begin{proof}[Proof of~\Cref{lemma:greedy-1}]
Let us assume without loss of generality that the features are already sorted decreasingly in terms of score, so \[s_1 \geq s_2 \geq \cdots \geq s_d.\]
This way, we have that $\vy^{(k)} \subseteq \vx$ is defined as follows:
\[ 
    y^{(k)}_i = \begin{cases}
        x_i & \text{if } i \leq k\\
        \bot & \text{otherwise}.
    \end{cases}
\]
The proof now requires considering two cases. First, if $s_{k+1} \geq 0$, then we can apply~\Cref{lemma:positive-score} to conclude that
\[ 
    \Pr_{\vz \sim \U(\vy^{(k+1)})}[\Lin(\vz) = \Lin(\vx)] \geq \Pr_{\vz \sim \U(\vy^{(k)})}[\Lin(\vz) = \Lin(\vx)].
\]
We will now show that if $s_{k+1} < 0$, then 
\[ 
    \Pr_{\vz \sim \U(\vy^{(k+1)})}[\Lin(\vz) = \Lin(\vx)] = 1,
\]
which will be enough to conclude. Indeed, as $s_{k+1} < 0$, we have that 
\[
   0 > s_{k+2} \geq s_{k+3} \geq \cdots \geq s_d, 
\]
from where we can repeatedly apply~\Cref{lemma:negative-score} to deduce 
\[ 
    \Pr_{\vz \sim \U(\vy^{(k+1)})}[\Lin(\vz) = \Lin(\vx)]  \geq \Pr_{\vz \sim \U(\vy^{(k+2)})}[\Lin(\vz) = \Lin(\vx)] \geq \cdots \geq \Pr_{\vz \sim \U(\vy^{(d)})}[\Lin(\vz) = \Lin(\vx)] = 1.
\]
This concludes the proof.
\end{proof}
\begin{claim}[Part 2 of~\Cref{lemma:greedy}]\label{lemma:greedy-2}
    Given a linear model $\Lin$, and an instance $\vx$, if $\vy^{(1)}, \ldots, \vy^{(d)}$ are the partial instances of $\vx$ such that $\vy^{(k)} \subseteq \vx$ is defined only in the top $k$ features of maximum score, then $\opt(\Lin, \vx, \delta) = k$ if and only if $\vy^{(k)}$ is a $\delta$-SR for $\vx$ but $\vy^{(k-1)}$ is not. 
\end{claim}

In order to prove~\Cref{lemma:greedy-2}, we will use a separate lemma. Let us define, for every $k \in [d]$ the set $P_k$ as the set of partial instances $\vy \subseteq \vx$ such that $\vy$ has $k$ defined features.
\begin{lemma}\label{lemma:greedy-3}
    For any $k \in [d]$, we have 
    \[
        \Pr_{\vz \sim \U(\vy^{(k)})}[\Lin(\vz) = \Lin(\vx)] = \max_{\vy \in P_k} \Pr_{\vz \sim \U(\vy)}[\Lin(\vz) = \Lin(\vx)].
    \]
\end{lemma}

Let us show immediately how~\Cref{lemma:greedy-2} can be proved using~\Cref{lemma:greedy-3}.
\begin{proof}[Proof of~\Cref{lemma:greedy-2}]
  For the forward direction, assume that $\opt(\Lin, \vx, \delta) = k$. Then, by definition, we have that there exists a $\delta$-SR $\vy^\star$ for $\vx$ such that $\vy^\star$ has $k$ defined features.
    By~\Cref{lemma:greedy-3}, we have that 
    \[ 
        \Pr_{\vz \sim \U(\vy^{(k)})}\left[\Lin(\vz) = \Lin(\vx)\right] \geq \Pr_{\vz \sim \U(\vy^\star)}\left[\Lin(\vz) = \Lin(\vx)\right] \geq \delta,
    \]
    and thus $\vy^{(k)}$ is a $\delta$-SR for $\vx$. On the other hand, if $\vy^{(k-1)}$ were to be a $\delta$-SR for $\vx$, then we would have $\opt(\Lin, \vx, \delta) \leq k-1$, a contradiction.
    For the backward direction, assume that $\vy^{(k)}$ is a $\delta$-SR for $\vx$ but $\vy^{(k-1)}$ is not. Then, by~\Cref{lemma:greedy-3}, we have that 
    \[
        \delta > \Pr_{\vz \sim \U(\vy^{(k-1)})}[\Lin(\vz) = \Lin(\vx)] = \max_{\vy \in P_{k-1}} \Pr_{\vz \sim \U(\vy)}[\Lin(\vz) = \Lin(\vx)],
    \]
    from where $\opt(\Lin, \vx, \delta) > k-1$, and because $\vy^{(k)}$ is a $\delta$-SR for $\vx$, we have $\opt(\Lin, \vx, \delta) \leq k$; we conclude that $\opt(\Lin, \vx, \delta) = k$.
\end{proof}

It thus only remains to prove~\Cref{lemma:greedy-3}.
\begin{proof}[Proof of~\Cref{lemma:greedy-3}]
Let $w_1, \ldots, w_d$ be the weights of $\Lin$, and $t$ its threshold. Let us use the $\oplus, \ominus$ notation defined in~\Cref{lemma:positive-score,lemma:negative-score}. 
   We will prove something slightly stronger than~\Cref{lemma:greedy-3}: that if $i$ and $j$ are features such that $s_i \leq s_j$, then for any partial instance $\vy$ such that $y_i \neq \bot$ and $y_j = \bot$, we have
    \[
        \Pr_{\vz \sim \U(\vy \ominus i \oplus j)}[\Lin(\vz) = \Lin(\vx)] \geq \Pr_{\vz \sim \U(\vy)}[\Lin(\vz) = \Lin(\vx)].
    \]
    If we prove this, then we can apply it repeatedly to deduce~\Cref{lemma:greedy-3}. To prove the claim, we start by defining
    \[ 
        S = \{ \ell \mid y_\ell \neq \bot \} \setminus \{i \},
    \]
    and 
    \[ 
        t' = t - \sum_{\ell \in S} y_\ell w_\ell.
    \]
   We will also assume without loss of generality that $\Lin(\vx) = 1$ since the other case is analogous. We can then rewrite the probabilities of interest as follows, using notation $\bar{S} := [d] \setminus S$:
   \[
    \Pr_{\vz \sim \U(\vy \ominus i \oplus j)}[\Lin(\vz) = \Lin(\vx)] = \Pr_{\vz \sim \U(\vy \ominus i \oplus j)}\left[\sum_{\ell \in \bar{S} \setminus \{i, j\}} z_\ell w_\ell  + x_j w_j + z_i w_i\geq t'\right],
   \]
   \[
    \Pr_{\vz \sim \U(\vy)}[\Lin(\vz) = \Lin(\vx)] = \Pr_{\vz \sim \U(\vy)}\left[\sum_{\ell \in \bar{S} \setminus \{i, j\}} z_\ell w_\ell  + x_i w_i + z_j w_j\geq t'\right].
   \]
    Let us write $t^\star = t' - \sum_{\ell \in \bar{S} \setminus \{i, j\}} z_\ell w_\ell$, and note that $t^\star$ is a random variable. With this notation, it remains to prove that 

    \begin{align*}
        &\Pr_{z_i, t^\star}[z_i w_i + x_j w_j \geq t^\star] \geq \Pr_{z_i, t^\star}[z_j w_j + x_i w_i \geq t^\star] \\
        \iff& \frac{1}{2}\Pr_{t^\star}[w_i + x_j w_j \geq t^\star] + \frac{1}{2}\Pr_{t^\star}[x_j w_j \geq t^\star] \geq \frac{1}{2}\Pr_{t^\star}[w_j + x_i w_i \geq t^\star] + \frac{1}{2}\Pr_{t^\star}[x_i w_i \geq t^\star] \\
        \iff& \Pr_{t^\star}[w_i + x_j w_j \geq t^\star] + \Pr_{t^\star}[x_j w_j \geq t^\star] \geq \Pr_{t^\star}[w_j + x_i w_i \geq t^\star] + \Pr_{t^\star}[x_i w_i \geq t^\star].
    \end{align*}
    We will prove the last inequality by cases, recalling that $s_j \geq s_i$ and thus $w_j (2x_j - 1) \geq w_i (2x_i - 1)$.
    \begin{itemize}
        \item (\textbf{Case 1:} $x_i = 1, x_j = 1$) The desired inequality is
        \begin{align*}
            \Pr_{t^\star}[w_i + w_j \geq t^\star] + \Pr_{t^\star}[w_j \geq t^\star] &\geq \Pr_{t^\star}[w_j + w_i \geq t^\star] + \Pr_{t^\star}[w_i \geq t^\star]\\
            \iff \Pr_{t^\star}[w_j \geq t^\star] &\geq \Pr_{t^\star}[w_i \geq t^\star],
        \end{align*}
        which is true since $s_j \geq s_i$ implies $w_j \geq w_i$ given $x_i = x_j = 1$.
        \item (\textbf{Case 2:} $x_i = 1, x_j = 0$) The desired inequality is
        \begin{align*}
            \Pr_{t^\star}[w_i \geq t^\star] + \Pr_{t^\star}[0\geq t^\star] &\geq \Pr_{t^\star}[w_j + w_i \geq t^\star] + \Pr_{t^\star}[w_i \geq t^\star]\\
            \iff \Pr_{t^\star}[0 \geq t^\star] &\geq \Pr_{t^\star}[w_j + w_i \geq t^\star],
        \end{align*}
        which is true since $s_j \geq s_i$ implies $-w_j \geq w_i$ given $x_i = 1, x_j = 0$, and thus $w_i + w_j \leq 0$.
        \item (\textbf{Case 3:} $x_i = 0, x_j = 1$) The desired inequality is 
        \begin{align*}
            \Pr_{t^\star}[w_i + w_j \geq t^\star] + \Pr_{t^\star}[w_j \geq t^\star] &\geq \Pr_{t^\star}[w_j \geq t^\star] + \Pr_{t^\star}[0 \geq t^\star],\\
            \iff \Pr_{t^\star}[w_i + w_j \geq t^\star] &\geq \Pr_{t^\star}[0\geq t^\star],
        \end{align*}
        which is true since $s_j \geq s_i$ implies $w_j \geq -w_i$ given $x_i = 0, x_j = 1$, and thus $w_i + w_j \geq 0$.
        \item (\textbf{Case 4:} $x_i = 0, x_j = 0$) The desired inequality is
        \begin{align*}
            \Pr_{t^\star}[w_i \geq t^\star] + \Pr_{t^\star}[0 \geq t^\star] &\geq \Pr_{t^\star}[w_j \geq t^\star] + \Pr_{t^\star}[0 \geq t^\star],\\
            \iff \Pr_{t^\star}[w_i \geq t^\star] &\geq \Pr_{t^\star}[w_j \geq t^\star],
        \end{align*}
        which is true since $s_j \geq s_i$ implies $-w_j \geq -w_i$ given $x_i = x_j = 0$, and thus $w_i \geq w_j$.
    \end{itemize}
\end{proof}
\Cref{lemma:greedy} now follows directly from~\Cref{lemma:greedy-1} and~\Cref{lemma:greedy-2}, and the sketch proof of~\Cref{thm:locally-minimal} can be completed as we now have proved~\Cref{lemma:positive-score} and~\Cref{lemma:negative-score}.



\end{document}
