\section{Computational Problem}

\cite{Waldchen_MacDonald_Hauch_Kutyniok_2021} showed that computing the smallest $\delta$-SR for arbitrary classifiers is hard for $\textrm{NP}^{\textrm{PP}}$, and that no algorithm can achieve an approximation factor (in terms of $k^\star$) of $d^{1-\alpha}$, $\alpha > 0$, unless $\ptime = \textrm{NP}$. Since then,~\cite{Arenas_Barcelo_Romero_Subercaseaux_2022} showed that even for the restricted class of decision trees, usually considered the interpretable, smallest $\delta$-SRs cannot be computed unless $\ptime = \textrm{NP}$, and furthermore,~\cite{Kozachinskiy_2023} proved that the approximation task is also hard for decision trees. Notably, none of these hardness results rely on the distribution $\D$ being complicated, as they were proved taking $\D = \textrm{Uniform}(\{0, 1\}^d)$. Let us formally define the computational problem at hand, assuming the uniform distribution.



 \csproblem{Uniform-Min-$\delta$-SR$(\mathcal{C})$}{a model $\M \in \mathcal{C}$, an instance $\vx$, a value $\delta \in [0, 1]$, and an integer $k$.}{\textsc{\bf Yes} if $k^\star(\M, \vx, \delta) \leq k$, and \textsc{\bf No} otherwise.}
 
Unfortunately, it turns out that even Uniform-Min-$\delta$-SR$(\textsc{Linear})$ is hard, as the amount
\[
    \Pr_{\vz \sim  \D(\vy)}\Big[\Lin(\vz) = 1\Big]
\]
cannot be computed in polynomial time unless $\ptime=\# \ptime$~\cite{NEURIPS2020_b1adda14}. 

\begin{proposition}
    Uniform-Min-$\delta$-SR$(\textsc{Linear})$ cannot be solved in polynomial time unless $\ptime = \# \ptime$.
\end{proposition}
\begin{proof}
    The proof is a simple modification of~\cite[Lemma 28]{NEURIPS2020_b1adda14}; let $(s_1, \ldots, s_n, T) \in \mathbb{N}^{n+1}$ be an instance of the $\# \ptime$-complete problem $\#\Knapsack$, that consists on counting the number of sets $S \subseteq \{s_1, \ldots, s_n\}$ such that $\sum_{s \in S}s \leq T$.  
    We can assume that $\sum_{i=1}^n s_i > T$, as otherwise the $\# \Knapsack$ instance is trivial.
    Then, let $\Lin$ be a linear model with weights $w_i = s_i$, and threshold $t = T+1$.
    Now, consider the problem of deciding whether $\#\Knapsack(s_1, \ldots, s_n, T) \geq k$, which cannot be solved in polynomial time unless $\ptime = \# \ptime$.
    Let $\vx = (1, 1, \ldots, 1)$, and $\delta = \frac{k}{2^{n}}$. We claim that $(\Lin, \vx, \delta, 0)$ is a Yes-instance for Uniform-Min-$\delta$-SR$(\textsc{Linear})$ if and only if $\#Knapsack(s_1, \ldots, s_n, T) \geq k$. 
    First, note that $\Lin(\vx) = 1$, since $\sum_{i=1}^n w_i x_i = \sum_{i=1}^n s_i \geq T+1 = t$. 
    For every set $S \subseteq \{s_1, \ldots, s_n\}$ such that $\sum_{s \in S}s \leq T$, its complement $\overline{S} := \{s_1, \ldots, s_n\} \setminus S$ holds $\sum_{s \in \overline{S}}s > T$, and as all values are integers, this implies as well
    \[
        \sum_{s \in \overline{S}} s \geq T+1 = t.
    \]
    To each such set $\overline{S}$, we associate the instance $\vz(\overline{S})$ defined as
    \[
        z(\overline{S})_i = \begin{cases}
            1 & \text{if } s_i \in \overline{S}\\
            0 & \text{otherwise}.
        \end{cases}
    \]
    Now note that 
    \[
        \Lin(\vz(\overline{S})) = \begin{cases}
            1 & \text{if } \sum_{s_i \in \overline{S}} w_i \geq T+1\\
            0 & \text{otherwise} \end{cases} = 1,
    \]
    and thus there is a bijection between the sets $S$ whose sum is at most $T$ and the instances $\vz(\overline{S})$ such that $\Lin(\vz(\overline{S})) = 1 = \Lin(\vx)$.
    To conclude, simply note that the previous bijection implies
    \[
        \Pr_{\vz \sim U(\{0, 1\}^n)}\Big[\Lin(\vz) = 1\Big] = \frac{\#\Knapsack(s_1, \ldots, s_n, T)}{2^n},
    \] 
    and thus $\vy$ has probability at least $\delta$ if and only if $\#\Knapsack(s_1, \ldots, s_n, T) \geq k$.
    

%     First, recall that the problem of counting the number of ``positive completions'' of a partial instance $\vy$ is $\# \ptime$-hard for linear models~\cite{NEURIPS2020_b1adda14}; that is, given a partial instance $\vy$, a linear model $\Lin$ and a positive integer $K$, it is $\mathrm{PP}$-hard to determine whether there are at least $K$ instances $\vz \in \comp(\vy)$ such that $\Lin(\vz) = 1$. That result follows from the hardness of $\sharpK$.
%    Moreover, in the proof of~\cite{NEURIPS2020_b1adda14} all weights are positive $\delta = \frac{V}{2^{n - |\vy|_\bot}}.$ 
\end{proof}

We will consider the problem of approximating the minimum size $\delta$-SRs for linear models, first under the uniform distribution, and then under product distributions. To do this, let us consider two senses in which one can approximate $\delta$-SRs.
\begin{itemize}
    \item \textbf{Approximation in terms of $\delta$}: Given a linear model $\Lin$, an instance $\vx$, and a value $\delta$, we want to compute a $\delta'$-SR of size $k^\star(\Lin, \vx, \delta')$ such that $\delta'$ is close to $\delta$. Intuitively, this corresponds to the idea of stakeholders not caring about the exact value of $\delta$; e.g., $90\%$ of completions of $\vy$ agree with $\vx$ has roughly the same human implications as $89.9997\%$.
    \item \textbf{Approximation in terms of $k^\star$}: Given a linear model $\Lin$, an instance $\vx$, and a value $\delta$, we want to compute a $\delta$-SR of size $k$ such that $k$ is not much bigger than $k^\star(\Lin, \vx, \delta)$. Intuitively, even though stakeholders want \emph{small} explanations, it is not required to find the smallest possible explanation.
\end{itemize}

We can now state an initial result.

\begin{theorem}\label{thm:delta-approximation}
    The Uniform-Min-$\delta$-SR$(\textsc{Linear})$ problem admits an FPRAS with respect to $\delta$. In particular, there exists an algorithm that computes a minimum $\delta'$-SR for some $\delta' \in [\delta, \delta + \epsilon]$, and succeeds with probability at least $1-\gamma$, running in time $\tilde{O}\left(   \frac{d^2}{\epsilon^2\gamma^2}\right)$.
\end{theorem}

% \begin{theorem}\label{thm:k-approximation}
%     The Uniform-Min-$\delta$-SR$(\textsc{Linear})$ problem can be approximated over $k^\star$ with additive error $1$ and probability of success $1-\gamma$ in time $\tilde{O}\left( \frac{d^2}{\epsilon^2\gamma^2}\right)$.
% \end{theorem}

In order to prove it we will need two basic ideas: first, the fact that we can estimate the probabilities of linear models accepting a partial instance through sampling, and second, that under the uniform distribution it is easy to decide which features ought to be part of small explanations.

\subsection{Estimating the Probability of Acceptance}
 The  hardness of computing
$\Pr_{\vz  \in \D(\vy)}[\M(\vz) = 1]$ is about computing it to arbitrarily high precision, i.e., with an additive error within $O(2^{-d})$. However, computing a less precise estimation of $\Pr_{\vz \in \D(\vy)}[\M(\vz) = 1]$ is simple, as the next fact (which is a direct consequence of Hoeffding's inequality)  states.

\begin{fact}\label{fact:hoeffding}
    Let $f$ be an arbitrary boolean function on $n$ variables. Let $M$ be any integer,
    and sample $M$ inputs $x_1, \ldots, x_M$ uniformly at random from $\{0, 1\}^n$. Then 
    \(
        \hat{\mu} := \frac{\sum_{i=1}^M [f(x_i) = 1]}{M}
    \)
    is an unbiased estimator for 
    \(
        \mu := \Pr_{x \in \{0, 1\}^n}[f(x) = 1],
    \)
    and 
    \[
    \Pr[\left|\hat{\mu} - \mu \right| \leq t] \geq 1 - 2e^{-2t^2 M},
    \]
    which is at least $1 - \gamma$ for $M = \frac{1}{2t^2} \log(2/\gamma)$.
\end{fact}

 \paragraph{Smoothed Explanations} As a consequence of the previous idea, although a minimum $\delta$-SR might be hard to compute, this crucially depends on the value of $\delta$. In the spirit of smoothed analysis, we define the computation of a min-$(\delta, \epsilon)$-SR as follows: first, a value $\delta^\star$ is chosen uniformly at random from $[\delta-\epsilon, \delta+\epsilon]$, and then a min-$\delta^\star$-SR is computed. Intuitively, the idea is that as $\delta^\star$ is chosen at random, it will be unlikely that a value that makes the computation hard is chosen. 

 We will prove the following proposition:

\begin{proposition}
    \label{prop:smoothed-explanation}
    Given a linear model $\Lin$ and an input $\vx$, we can compute a $(\delta, \epsilon)$-SR successfully with probability $1 - \gamma$ in time polynomial in $d$, $1/\epsilon$ and $1/\gamma$. In particular, in time $\tilde{O}\left( \frac{d^2}{\epsilon^2\gamma^2}\right)$.
\end{proposition}

Note that~\Cref{prop:smoothed-explanation} immediately gives us~\Cref{thm:delta-approximation}, as we can set $\delta' =  \delta+\epsilon/2$ and $\epsilon' = \epsilon/2$, which means with probability $1 - \gamma$ we will obtain a $(\delta', \epsilon')$-SR, whose probability guarantee is in 
\[
  [\delta' - \epsilon', \delta' + \epsilon'] =  [\delta, \delta + \epsilon].
\]

Before proving~\Cref{prop:smoothed-explanation}, we need to prove a lemma concerning the easiness of selecting the features of the desired explanation.

\subsection{Feature Selection}

Even if we were granted an oracle computing the probabilities
\(
    \Pr_{\vz  \in \D(\vy)}[\M(\vz) = 1]
\), that would not be necessarily enough to efficiently compute the minimum $\delta$-SR. Indeed, for decision trees, the counting problem can be easily solved in polynomial time~\cite{NEURIPS2020_b1adda14}, and yet the computation of $\delta$-SRs of minimum size is hard, even to approximate~\cite{Arenas_Barcelo_Romero_Subercaseaux_2022,Kozachinskiy_2023}.
Intuitively, the problem for decision trees is that, even if we were told that the minimum $\delta$-SR has exactly $k$ features, it is not obvious how to search for it better than enumerating all $\binom{d}{k}$ subsets. The case of linear models, however, is different, at least under the uniform distribution. In this case, every feature $i$ that is not part of the explanation will take value $0$ or $1$ independently with probability $\nicefrac{1}{2}$, and contribute to the classification according to its weight $w_i$. In other words, we can sort the features according to their weights (with some care about signs), and select them greedily to build a small $\delta$-SR. A proof for the deterministic case ($\delta = 1$) was already given in~\cite{NEURIPS2020_b1adda14} and sketched earlier on by~\cite{ExplainingNaiveBayes}.

\begin{definition}
    Given a linear model $\Lin = (\vw, t)$, and an instance $\vx$, both having dimension $d$, we define the \emph{score} of feature $i \in [d]$ as 
    \[
        s_i := w_i \cdot (2x_i - 1) \cdot (2\left(\Lin(\vx) - 1\right)).    
    \]
\end{definition}

In other words, the sign of $s_i$ is $+1$ if the feature is ``helping'' the classification, and $-1$ if it is ``hurting'' it. The magnitude of $s_i$ is proportional to the weight of the feature $i$.
For the uniform distribution, we can prove the following lemma that basically states that, if we are promised a $\delta$-SR of size $k$ exists, then one can be found by taking the top-$k$ features of maximum score $s_i$.

\begin{lemma}\label{lemma:greedy}
Given a linear model $\Lin$, an instance $\vx$,  a value $\delta \in (0, 1]$, and an integer $k$, then there exists a $\delta$-SR for $\vx$ under $\Lin$ of size $k$ if and only if the partial instance defined only on the $k$ features of $\vx$ with maximum score is a $\delta$-SR for $\vx$ under $\Lin$.
\end{lemma}

\todo[inline]{Bernardo: The proof of this is trivial but annoying, it suffices to do an exchange argument. I will write it down later.}

With this lemma in hand, we can now prove~\Cref{prop:smoothed-explanation}.

\renewcommand{\algorithmiccomment}[1]{\; \; \texttt{/* #1 */}}
\begin{algorithm}[tb]
	\caption{Uni-Linear MonteCarlo}
	\label{alg:algorithm}
	\textbf{Input}: Linear model $\Lin$, instance $\vx$, parameters $\delta \in (0, 1)$\\
	\textbf{Parameter}:  $\varepsilon \in (0, 1)$,  $\gamma \in (0, 1)$\\
	\textbf{Output}: A value $\delta^\star \in [\delta-\varepsilon, \delta+\varepsilon]$ together with an optimal $\delta^\star$-SR explanation for $\vx$.\\
	\begin{algorithmic}[1] %[1] enables line numbers
	\STATE Sample $\delta^\star$ uniformly at random from $[\delta-\varepsilon, \delta + \varepsilon]$
	\STATE Compute the score $s_i$ for each feature
	\STATE Let $\vec{\mathcal{F}}$ be the sequence of pairs $(i, s_i)$ sorted decreasingly by $s_i$
	\STATE Let $\ell$ be the max value such that $\vec{\mathcal{F}}[\ell] = (\lambda, s_\lambda)$ has $s_\lambda > 0$.
	\STATE For $k \in \{1, \ldots, \ell\}$, let $\vy^{(k)} \subseteq \vx$ be the partial instance defined only in features $\vec{\mathcal{F}}[1][1], \ldots, \vec{\mathcal{F}}[k][1]$.
	\STATE Let $\textrm{LB} = 0$, and $\textrm{UB} = \ell$.
	\STATE $M \gets \frac{2\log d^2}{\varepsilon^2 \gamma^2} \log(2 \log d / \gamma)$.

	\WHILE[Binary Search]{$\textrm{LB} \neq \textrm{UB}$} 
	% \COMMENT{Binary Search}
		\STATE $m \gets \left(\textrm{LB} + \textrm{UB} \right)/2$.
		\STATE $\hat{v}_m \gets \textsc{MonteCarloSampling}(\Lin, \vy^{(m)}, M)$.
		\IF {$\hat{v}_m \geq \delta^\star$}
			\STATE $\textrm{UB} \gets m$
		\ELSE
			\STATE $\textrm{LB} \gets (m+1)$
		\ENDIF
	\ENDWHILE
	\STATE $k^\star \gets \textrm{LB}$ (or equivalently, $\textrm{UB}$)
	\STATE \RETURN $(\delta^\star, \vy^{(k^\star)})$
	\end{algorithmic}
	\end{algorithm}


\begin{algorithm}[tb]
	\caption{MonteCarloSampling}	\label{alg:montecarlo}
	\textbf{Input}: Linear model $\Lin$, instance $\vx$, number of samples $M$\\
	\textbf{Output}: An estimate~$\hat{v}$~of~$\Pr_{\vz \in \mathcal{U}_\vx}[\Lin(\vz)=1]$.\\
	\begin{algorithmic}[1]
	\STATE $\hat{v} \gets 0$
	\FOR{$i = 1$ to $M$}
		\STATE Sample $\vz \sim \textsc{Comp}(\vx)$
		\STATE $\hat{v} \gets \hat{v} + \Lin(\vz)$
	\ENDFOR
	\STATE $\hat{v} \gets \hat{v}/M$
	\STATE \RETURN $\hat{v}$
\end{algorithmic}
\end{algorithm}


	

\begin{proof}[Proof of~\Cref{prop:smoothed-explanation}]
We use~\Cref{alg:algorithm}. Let us define the values $v_k$ as
	\[
		v_k := \Pr_{\vz \in \textsc{Comp}(\vy^{(k)})}[\Lin(\vz) = 1].
	\]
As a result of the binary search (lines 8-16), the algorithm obtains $k^\star$, the smallest $k$ such that
\[
	\hat{v}_k \geq \delta^\star,
\]
and our goal is to show that with good probability $k^\star$ is also the smallest $k$ such that $v_k \geq \delta^\star$, which would imply the correctness of the algorithm.
Let $S$ be a random variable corresponding to the set of values $k$ such that~\Cref{alg:algorithm} enters line $10$ with $m=k$, and note that if for every $k$ in $S$ it happens that the events 
\[
	A_k := \left(v_k \geq \delta^\star \right) \text{ and }  B_k := \left(\hat{v_k}(M) \geq \delta^\star\right)
\]
are equivalent (i.e., either both occur or neither occurs), then the algorithm will succeed, as that would indeed imply that $k^\star$ is the smallest $k$ such that $v_k \geq \delta^\star$. 

Then, define events $E_k$ and $F_k$ as follows:
\begin{align*}
	E_k &:= |\delta^\star - v_k| > \frac{\epsilon \gamma}{\log d},\\
	F_k &:= k \in S \land |\hat{v_k}(M) - v_k| \leq \frac{\epsilon \gamma}{\log d}.
\end{align*}
We claim that if both $E_k$ and $F_k$ hold for some $k$, then $A_k$ and $B_k$ are equivalent events. Indeed,
\begin{align*}
	A_k &\iff v_k \geq \delta^\star\\
		&\iff v_k \geq \delta^\star + \frac{\epsilon \gamma}{\log d} \tag{By $E_k$}\\
		&\iff v_k - \frac{\epsilon \gamma}{\log d} \geq \delta^\star\\
		&\iff \hat{v_k}(M) \geq \delta^\star \tag{By $F_k$}\\
		&\iff B_k.
\end{align*}

Thus, if we show that $E_k$ and $F_k$ hold with good probability for every $k \in S$, we can conclude the theorem.
To see that, note first that for every $k \in [d]$, line 1 implies
\[
	\Pr[\overline{E_k}] = \Pr\left[\delta^\star \in \left[v_k - \frac{\epsilon \gamma}{\log d}, v_k + \frac{\epsilon \gamma}{\log d}\right]\right] \leq \frac{\frac{2\epsilon \gamma}{\log d}}{2\epsilon} = \frac{\gamma}{\log d}.
\]

It is tempting to say that we have 
\[ 
	\Pr\left[\bigcap_{k \in S} E_k\right] = 1 -  \Pr\left[\bigcup_{k \in S} \overline{E_k} \right] \geq 1 - \gamma,
\]
through a union bound on the $\log d$ elements of $S$\footnote{For simplicity we will say $|S| \leq \log d$, even though the exact bound for a binary search is $|S| \leq \lfloor \log d  + 1 \rfloor$.}. However, $S$ itself is a random variable, and $E_k$ and $k \in S$ are not (necessarily) independent events, so we need to be more careful. 
Using the law of total probabilities, we have 
\begin{align*}
	\Pr\left[\bigcap_{k \in S} E_k\right] &= \sum_{S' \subseteq [d]} \Pr\left[S = S'  \mid \bigcap_{k \in S'} E_k \right] \Pr\left[\bigcap_{k \in S'} E_k\right]\\
	&= \sum_{\substack{S' \subseteq [d]\\ |S'| \leq \log d}} \Pr\left[S = S' \mid \bigcap_{k \in S'} E_k\right] \Pr\left[\bigcap_{k \in S'} E_k\right],
\end{align*}
where we can now effectively use the union bound to say that for any fixed $S'$ with $|S'| \leq \log d$ we have
\[ 
	\Pr\left[\bigcap_{k \in S'} E_k \right] \geq 1 - \gamma,
\]
from where we get
\begin{align*}
	\Pr\left[\bigcap_{k \in S} E_k\right] &=  \sum_{\substack{S' \subseteq [d]\\ |S'| \leq \log d}} \Pr\left[S = S'\mid \bigcap_{k \in S'} E_k\right] \Pr\left[\bigcap_{k \in S'} E_k\right]\\
	&\geq (1-\gamma) \sum_{\substack{S' \subseteq [d]\\ |S'| \leq \log d}} \Pr\left[S = S' \mid \bigcap_{k \in S'} E_k\right]\\
	&= 1 - \gamma.
\end{align*}
 Let us now argue that $F_k$ holds with good probability for every $k \in S$. Indeed, we have for any $k$ that
 \begin{align*}
 \Pr\left[ \, \overline{F_k} \mid k \in S \right] &= \Pr\left[|\hat{v_k}(M) - v_k| > \frac{\epsilon \gamma}{\log d}\right]\\
 	&\leq \frac{\gamma}{\log d}, \tag{By~\Cref{fact:hoeffding} and line 7.}
 \end{align*}
from where a final union bound (using the same trick as for $E_k$) yields 
\begin{align*}
	\Pr\left[\bigcap_{k \in S} F_k\right] &= 1 - \Pr\left[\bigcup_{k \in S} \overline{F_k}\right]\\
	&\geq 1 - \sum_{k \in S} \Pr\left[\overline{F_k} \mid k \in S\right]\\
	&\geq 1 - \sum_{k \in S} \frac{\gamma}{\log d}\\
	&= 1 - \gamma.
\end{align*}
Therefore, the algorithm will succeed with probability at least 
\[ 
	\Pr\left[\bigcap_{k \in S} E_k\right] \cdot \Pr\left[\bigcap_{k \in S} F_k\right] \geq (1-\gamma)^2 \geq 1-2\gamma.
\]
The runtime is simply
$O(\log d \cdot M \cdot d  )$; as (i) the binary search performs $O(\log d)$ steps; (ii) each of the binary search steps requires $M$ samples, and (iii) each sample requires evaluating the model $\Lin$ and thus takes time $O(d)$. Naturally, running the algorithm with $\gamma' = 1/2 \cdot \gamma$ will yield a success probability of $1-\gamma$ without changing the asymptotic runtime, and thus we conclude the proof.


% We thus want to lower bound the probability of some decently large value $\Delta$ such that $|\delta^\star - v_k| \geq \Delta$ for every $k \in [d]$; that way, it will be enough to estimate the values $v_k$ up to an additive error of $\Delta/2$. Given that there are $d$ values $v_
% k$, and each of them forbids a segment of size $2\Delta$ for $\delta^\star$, then the probability of choosing a value of $\delta^\star$ such that $|\delta^\star - v_k| \geq \Delta$ for every $k \in [d]$ is at least
% \[
% 	1 - \frac{d\Delta}{\epsilon}.
% \]
% As we want our algorithm to succeed with probability $1 - \gamma$, that requires 
% \(
% \frac{d\Delta}{\epsilon} \leq \gamma
% \)
% and thus
% \[
% 	\Delta \leq \frac{\epsilon \gamma}{d}.
% \]
% Thus we need to sample with an additive error smaller than
% \(
% \frac{\epsilon}{2d \gamma},
% \)
% which requires the number of samples $M$ to be such that
% \[
% 	\frac{\log M}{\sqrt{M}} \leq \frac{\epsilon\gamma}{2d}.
% \] 
% %%Considering that
% %%\[
% %%	\frac{\log M}{\sqrt{M}} \leq \frac{\log M}{\log M \sqrt[2+\gamma]{M}} = \frac{1}{\sqrt[2+\gamma]{M}},
% %%\]
% %Let us try to simplify things by considering establishing
% %\[
% %	\frac{1}{\sqrt{M'}} \leq \frac{\epsilon\gamma}{2d}, 
% %\]
% %or equivalently,
% %\[
% %	M \geq \left( \frac{2d}{\epsilon\gamma} \right)^2.
% %\]
% Now we pose
% \[
% 	M =	9\cdot \left( \frac{2d}{\epsilon\gamma} \right)^2\log\left(\frac{2d}{\epsilon\gamma}\right)^2,
% \]
% so that $\log M \leq 3\log\left(\frac{2d}{\epsilon\gamma}\right)$, and thus
% \[
% 	\frac{\log M}{\sqrt{M}} \leq \frac{3\log(2/(\epsilon\gamma))}{3\left( \frac{2d}{\epsilon\gamma} \right)\log\left(\frac{2d}{\epsilon\gamma}\right)} = \frac{\epsilon\gamma}{2d},
% \]
% as desired.

% Thus, the algorithm above, which requires a sampling process at most $\log d$ times, using binary search to find the min, can be carried out with probability of success at least $1-\gamma$. in time 
% \[
% O(\log d \cdot M) = \tilde{O}\left(\left(\frac{d}{\epsilon\gamma}\right)^2\right).
% \]


% \begin{figure}
% 	\centering
% 	\scalebox{0.8}{
% 	\begin{tikzpicture}
% 		\node[] (beg) at (0, 0) {};
% 		\node[] (end) at (11, 0) {};
% 				\node[] (a) at (10, 0.5) {\small $1$};
% 		\node[] (amark) at (10, 0) {\small $\mid$};
% 		\node[] (b) at (5, 0.5) {\small $\nicefrac{1}{2}$};
		
% 		\node[] (b) at (0, 0.5) {\small $0$};
		
% 		\node[] (bmark) at (5, 0) {\small $\mid$};
		
% 		\node[] (leftPar) at (5.5, 0) {\textcolor{blue}{$\Big[$}};
% 		\node[] (rightPar) at (9.5, 0) {\textcolor{blue}{$\Big]$}};
		
% 		\draw[-, blue, dashed, very thick]  (leftPar) -- (rightPar);
		
% 		\draw[|-, very thick] (beg) -- (leftPar);
% 			\draw[->, very thick] (rightPar) -- (end);

		
% 		\node[] (left) at (5.5, -0.7) {\textcolor{blue}{\scriptsize $\delta-\varepsilon$}};
% 		\node[] (right) at (9.5, -0.7) {\textcolor{blue}{\scriptsize $\delta+\varepsilon$}};
		
% 		\node[] (delta) at (7.5, -0.5) {\textcolor{blue}{$\delta$}};
		
% 		\node[] (dmark) at (7.5, 0) {\textcolor{blue}{\small $\mid$}};
		
		
% 		\node[] (v1) at (8.8, 0) {\textcolor{purple}{\small $\mid$}};
% 		\node[] (v1) at (8.8, -0.4) {\textcolor{purple}{\scriptsize $v_1$}};
		
% 		\node[] (v2) at (6.6, 0) {\textcolor{purple}{\small $\mid$}};
% 		\node[] (v2) at (6.6, -0.4) {\textcolor{purple}{\scriptsize $v_2$}};
		
% 		\node[] (v4) at (3.6, 0) {\textcolor{purple}{\small $\mid$}};
% 		\node[] (v4) at (3.6, -0.4) {\textcolor{purple}{\scriptsize $v_3$}};
		
% 		\node[] (v3) at (1.2, 0) {\textcolor{purple}{\small $\mid$}};
% 		\node[] (v3) at (1.2, -0.4) {\textcolor{purple}{\scriptsize $v_4$}};
		
% 		\draw[rectangle, fill=purple, line width=0pt] (6.3, -0.2) -- (6.3, 0.2) -- (6.9, 0.2) -- (6.9, -0.2) -- (6.3, -0.2);
		
% 		\draw[rectangle, fill=purple, line width=0pt] (8.5, -0.2) -- (8.5, 0.2) -- (9.1, 0.2) -- (9.1, -0.2) -- (8.5, -0.2);
		
		
% 		\node at (6.6, 0.65) {$\Delta$};
% 		\node at (8.8, 0.65) {$\Delta$};
% 		\draw[decorate, decoration={brace, amplitude=3pt}, thick] (6.3, 0.28) -- (6.9, 0.28);
		
% 			\draw[decorate, decoration={brace, amplitude=3pt}, thick] (8.5, 0.28) -- (9.1, 0.28);
		
% 		\node[] (deltastar) at (8.1, -1.2) {\large \textcolor{violet}{$\delta^\star$}};
% 		\draw[-, , violet, thick] (8.0, -0.9) -- (8.0, 0.9);

 
% 	\end{tikzpicture}
% 	}
% 	\caption{Illustration of the proof, using $d=4, \delta = \nicefrac{3}{4}$ as an example. In this case, the probability of failure is $\frac{2\Delta}{\varepsilon}.$}
% \end{figure}

\end{proof}


% Let us restate and prove our other theorem.

% \begin{theorem}
%     The Uniform-Min-$\delta$-SR$(\textsc{Linear})$ problem can be approximated over $k^\star$ with additive error $1$ and probability of success $1-\gamma$ in time $\tilde{O}\left( \frac{d^2}{\epsilon^2\gamma^2}\right)$.
% \end{theorem}
% \begin{proof}[Proof sketch]
%     Sort the features according to \emph{score} again, then we try to find the minimum $k$ such that the top-$k$ most important features give a probability of $\delta$. In particular, we will accept the first $k$ such that the empirical probability is above $\delta + (1-\delta)/2d$. The chances of the ``true'' probability being above $\delta$ will naturally be good, the problem is to guarantee that we won't overshoot by much. The main insight here is that, if $k^*$ is the true optimal answer (thus having probability at least $\delta$), then $k^*+1$ should always have the desired probability; note that the remaining probability is $1-\delta$, and there are $d-{k^*}$ features remaining, so the next most important one must yield at least a gain of 
%     $\frac{1-\delta}{d - k^*} \geq \frac{1 - \delta}{d}$. So by sampling with an error of $\frac{1-\delta}{2d}$ we will recognize this, which by using the fact requires 
% $
%     \tilde{O}\left(\left(\frac{1-\delta}{2d}\right)^2\right)
% $ many samples to ensure with high probability. Note that the problematic case is now when $\delta \approx 1$. 
% \end{proof}